\documentclass[%
master,      % тип документа
natbib,      % использовать пакет natbib для "сжатия" цитирований
subf,        % использовать пакет subcaption для вложенной нумерации рисунков
href,        % использовать пакет hyperref для создания гиперссылок
colorlinks,  % цветные гиперссылки
%fixint,     % включить прямые знаки интегралов
]{disser}

\usepackage[
a4paper, mag=1000,
left=3cm, right=1cm, top=2cm, bottom=2cm, headsep=0.7cm, footskip=1cm
]{geometry}

\usepackage{natbib}
\usepackage[intlimits]{amsmath}
\usepackage{amssymb,amsfonts}

\usepackage[T2A]{fontenc}
\usepackage[utf8]{inputenc}
\usepackage[english,russian]{babel}
\ifpdf\usepackage{epstopdf}\fi
\usepackage[autostyle]{csquotes}

\usepackage[singlelinecheck=false]{caption}
\usepackage{cmap}
\usepackage{cancel}

%\usepackage[document]{ragged2e}
\pretolerance=10000
%\sloppy

% Шрифт Times в тексте как основной
%\usepackage{tempora}
% альтернативный пакет из дистрибутива TeX Live
\usepackage{cyrtimes}

% Шрифт Times в формулах как основной
%\usepackage[varg,cmbraces,cmintegrals]{newtxmath}
% альтернативный пакет
\usepackage[lite]{mtpro2}
\pdfmapfile{mtpro2.map}

% Список сокращений и условных обозначений
\usepackage[intoc,nocfg,russian]{nomencl}
\newcommand{\nomencl}[2]{#1 --- #2\nomenclature{#1}{#2}}
\setlength{\nomlabelwidth}{3em}
\setlength{\nomitemsep}{-\parsep}
\renewcommand{\nomlabel}[1]{#1 ---}
\makenomenclature

% Плавающие рисунки "в оборку".
\usepackage{wrapfig}

% Номера страниц снизу и по центру
\pagestyle{footcenter}
\chapterpagestyle{footcenter}

% Точка с запятой в качестве разделителя между номерами цитирований
%\setcitestyle{semicolon}

% Использовать полужирное начертание для векторов
\let\vec=\mathbf

% Включать подсекции в оглавление
\setcounter{tocdepth}{2}

% Подключить пакет для вставки кода в файл
\usepackage{listings}

\definecolor{codegreen}{rgb}{0,0.6,0}
\definecolor{codegray}{rgb}{0.5,0.5,0.5}
\definecolor{codepurple}{rgb}{0.58,0,0.82}
\definecolor{backcolour}{rgb}{0.95,0.95,0.92}

\lstset{ %
language=Python,                 % выбор языка для подсветки (здесь это С)
%basicstyle=\small\sffamily, % размер и начертание шрифта для подсветки кода
basicstyle=\tiny, % размер и начертание шрифта для подсветки кода
numbers=left,               % где поставить нумерацию строк (слева\справа)
%numberstyle=\tiny,           % размер шрифта для номеров строк
stepnumber=1,                   % размер шага между двумя номерами строк
numbersep=5pt,                % как далеко отстоят номера строк от подсвечиваемого кода
backgroundcolor=\color{white}, % цвет фона подсветки - используем \usepackage{color}
commentstyle=\color{codegreen},
keywordstyle=\color{magenta},
numberstyle=\color{codegray},
stringstyle=\color{codepurple},
showspaces=false,            % показывать или нет пробелы специальными отступами
showstringspaces=false,      % показывать или нет пробелы в строках
showtabs=false,             % показывать или нет табуляцию в строках
frame=single,              % рисовать рамку вокруг кода
tabsize=2,                 % размер табуляции по умолчанию равен 2 пробелам
captionpos=t,              % позиция заголовка вверху [t] или внизу [b]
breaklines=true,           % автоматически переносить строки (да\нет)
breakatwhitespace=false, % переносить строки только если есть пробел
escapeinside={#}   % если нужно добавить комментарии в коде
}

\usepackage{graphicx}
\graphicspath{{pictures/}}

\captionsetup[figure]{justification=centering}
\captionsetup[table]{position=top,aboveskip=5pt}

% Покдлючить пакет с возможностью подсчета станиц, рисунков, формул, таблиц и т.д.
\usepackage{totcount}
\regtotcounter{page}
\newtotcounter{mfigure}
\newtotcounter{mequation}
\newtotcounter{mtable}
\newtotcounter{bibcnt}

%% Нумерация с 4 страницы
%\setcounter{page}{4}

\def\oldfigure{} \let\oldfigure=\figure
\def\figure{\stepcounter{mfigure}\oldfigure}

\def\oldequation{} \let\oldequation=\equation
\def\equation{\stepcounter{mequation}\oldequation}

\def\oldmultline{} \let\oldmultline=\multline
\def\multline{\stepcounter{mequation}\oldmultline}

\def\oldtable{} \let\oldtable=\table
\def\table{\stepcounter{mtable}\oldtable}

\def\oldbibitem{} \let\oldbibitem=\bibitem
\def\bibitem{\stepcounter{bibcnt}\oldbibitem}

\makeatletter
\setlength{\parindent}{7.5ex}
\renewcommand{\tocprethechapter}{}
\renewcommand{\tocpostthechapter}{}
\renewcommand{\tocpostthesection}{}
\renewcommand{\postthesection}{\ }
\renewcommand{\tocsectionindent}{0em}
\renewcommand{\tocsectionnameindent}{1.5em}
\renewcommand{\sectionindent}{10em}
\renewcommand{\captionlabeldelim}{\textendash}
\renewcommand{\thetable}{\@arabic\value{mtable}}
\renewcommand{\theequation}{\@arabic\value{mequation}}
\renewcommand{\fnum@figure}{Рисунок \thefigure}
\renewcommand{\labelitemi}{\textendash}
\makeatother

\DeclareCaptionFormat{plain}{\captionlabelfont #1 \captionlabeldelim \ #3 \par}

\makeatletter
\renewcommand*\l@chapter[2]{%
\ifnum \c@tocdepth >\m@ne
%\addpenalty{-\@highpenalty}
%\vskip 1.0em \@plus\p@
%\setlength\@tempdima{4.5em}
\begingroup
\parindent \z@ \rightskip \@pnumwidth
\parfillskip -\@pnumwidth
%\leavevmode\tocchapterfont
\advance\leftskip\@tempdima
\hskip -\leftskip
#1\nobreak
\tocchapterfillfont\tocchapterfill\hfill
\nobreak\hb@xt@\@pnumwidth{\hss\tocchapternumfont #2}\par
\penalty\@highpenalty
\endgroup
\fi}
\makeatother

\makeatletter
\renewcommand{\@makechapterhead}[1]{% Начало макроопределения
%\vspace*{20 pt}% Пустое место вверху страницы
{\parindent=0pt
\hspace{1.5 cm}
\upshape \mdseries \rmfamily \normalsize \large
\thechapter \hspace{2 pt} #1\par % заголовок главы
\nopagebreak
\vspace{5 pt}
}
}
\makeatother

\renewcommand{\sectionfont}{\upshape \mdseries \rmfamily \normalsize \large}
\renewcommand{\sectionindent}{1.6 cm}

\begin{document}

    % Переопределение стандартных заголовков
    \def\introname{\upshape \mdseries \rmfamily \normalsize ВВЕДЕНИЕ}
    \def\contentsname{\upshape \mdseries \rmfamily \normalsize СОДЕРЖАНИЕ}
    \def\conclusionname{\upshape \mdseries \rmfamily \normalsize ЗАКЛЮЧЕНИЕ}
    \def\bibname{\upshape \mdseries \rmfamily \normalsize СПИСОК ИСПОЛЬЗОВАННЫХ ИСТОЧНИКОВ}
    \def\defsname{\upshape \mdseries \rmfamily \normalsize СПИСОК СОКРАЩЕНИЙ И УСЛОВНЫХ ОБОЗНАЧЕНИЙ}

    %
    % Титульный лист на русском языке
    %

%    \institution{Название организации}
%
%    % Имя лица, допускающего к защите (зав. кафедрой)
%    \apname{ФИО зав. кафедрой}
%
%    \title{ДИССЕРТАЦИЯ\\[-14pt]на соискание ученой степени\\МАГИСТРА}
%
%    \topic{Тема диссертации}
%
%    % Автор
%    \author{ФИО автора}
%    % Группа
%    \group{1111/1}
%    % Номер направления
%    \coursenum{111111}
%    \course{Название направления}
%    % Номер магистерской программы
%    \masterprognum{111111}
%    \masterprog   {Название программы}
%
%    % Научный руководитель
%    \sa      {ФИО руководителя}
%    \sastatus{д.~ф.-м.~н., ст.~н.~с.}
%    % Второй научный руководитель
%    %\sasnd      {ФИО руководителя}
%    %\sasndstatus{д.~ф.-м.~н., ст.~н.~с.}
%
%    % Рецензент
%    \rev      {ФИО рецензента}
%    \revstatus{д.~ф.-м.~н., в.~н.~с.}
%    % Второй рецензент
%    %\revsnd      {ФИО рецензента}
%    %\revsndstatus{д.~т.~н., ст.~н.~с.}
%
%    % Консультант
%    \con{ФИО консультанта}
%    \conspec{вопросам\\охраны труда}
%    \constatus{к.~т.~н., доц.}
%    % Второй консультант
%    %\consnd{ФИО консультанта}
%    %\consndspec{экономическим\\вопросам}
%    %\consndstatus{к.~э.~н., доц.}
%
%    % Город и год
%    \city{Санкт-Петербург}
%    \date{\number\year}
%
%    \maketitle

    %%
    %% Titlepage in English
    %%
    %
    %\institution{Name of Organization}
    %
    %% Approved by
    %\apname{Professor S.\,S.~Sidorov}
    %
    %\title{Master's Thesis}
    %
    %% Topic
    %\topic{Dummy Title}
    %
    %% Author
    %\author{Author's Name} % Full Name
    %\course{Physics} % Specialization
    %
    %\group{} % Study Group
    %\masterprog   {Title of program}
    %
    %% Scientific Advisor
    %\sa       {I.\,I.~Ivanov}
    %\sastatus {Professor}
    %
    %% Reviewer
    %\rev      {P.\,P.~Petrov}
    %\revstatus{Associate Professor}
    %
    %% Consultant
    %\con{}
    %\conspec{}
    %\constatus{}
    %
    %% City & Year
    %\city{Saint Petersburg}
    %\date{\number\year}
    %
    %\maketitle[en]

    % Реферат для НИР
%    %! Author = jaroslav
%! Date = 22.04.20
%\begin{titlepage}
%    \tiny
%    \scriptsize
%    \footnotesize
%    \small
%    \normalsize
%    \large
%    \Large
%    \LARGE
%    \huge
%    \Huge
%    \begin{center}
%        \large
%        РЕФЕРАТ
%    \end{center}
%
%    Отчет \total{page} страницы, %\total{mequation} формул,
%    \total{mfigure} рисунков,
%    \total{mtable} таблиц, \total{bibcnt} источников,
%
%    \hspace{0.5cm}ТЕОРИЯ ПРИНЯТИЯ РЕШЕНИЙ, ЭКСПЕРИМЕНТ ЭЛЛСБЕРГА,
%
%    \hspace{0.5cm}ОШИБКА КОНЪЮНКЦИИ, ЭФФЕКТ ДИЗЪЮНКЦИИ,
%
%    \hspace{0.5cm}ИНТЕРНЕТ РЕСУРСЫ, ДИЛЕММА ЗАКЛЮЧЕННОГО
%
%    Объектом исследования является модель социально-значимых Интернет
%    ресурсов, которое рассматривает отношения между индивидами в рамках
%    дискусси на популярном Интернет площадке.
%
%    Целью данной работы является поиск связей между моделью социально-
%    значимых Интернет ресурсов и экспериментами, объяснение которых
%    простыми моделями не представляется возможным.
%    Выявление этих связей позволит узнать возможные пути объяснения
%    этих эффектов, а также предсказать возможные другие когнитивные
%    эффекты.
%
%    В результате работы была вывлена связь между экспериментами и
%    моделью социально-значимых Интернет ресурсов
%
%    В дальнейшем, эта работа позвоит выявить связь между моделью
%    социально-значимых Интернет ресусов и квантово-подобным моделями
%    приянтия решений в условиях неопределенности
%\end{titlepage}


    % Содержание
    \tableofcontents

    % Введение
    %! Author = jaroslav
%! Date = 22.04.20
\intro

Современная наука хорошо справляется с естественнонаучными задачами окружающего мира.
Разработаны теории, модели которые хорошо описывают основные законы физики, химии, биологии.
И только к началу двадцатого века, ученые пришли к решению когнитивых задач.
Конечно, ранее эти задачи тоже изучались, но не так интенсивно и обсуждались лишь в
кругах филосовских мыслителей.
Такое позднее изучение когнитивных задач связано, конечно же, со сложностью формулировки,
постановкой экспериментов и малой применимостью результатов.
В современное время, необходимость в изучении когнитивных наук стало весомее из-за различных
эффектов с ошибками выбора правильного решения, развития задач искусственного интелекта,
изменений социальных отношений в современном обществе.
Как и любая другая наука, когнитивные задачи имеют свои парадоксы, которые сложно объяснить
простыми словами или законами.
В таком случае необходимо подбирать решения используя уже известные науке интрументы.
В данной работе такими инструментами являются теория графов и математический формализм.

    % Глава 1
    \input{chapters/overview.tex}
    % Глава 2
    %! Author = jaroslav
%! Date = 15.04.20
%%%%%%%%%%%%%%%%%%%%%%%%%%%%%%%%%%%%%%%%%%%%%%%%%%%%%%%%%%%%%%%%%%%%%%%%%%%%%%%%%%%%%%%%%%%%%%%%%%%%
%%%%%%%%%%%%%%%%%%%%%%%%%%%%%%%%%%%%%%%%%%%%%%%%%%%%%%%%%%%%%%%%%%%%%%%%%%%%%%%%%%%%%%%%%%%%%%%%%%%%
\section{Модель социально-значимых Интернет ресурсов}

Прежде чем переходить к рассмотрению квантово-подобных моделей принятия решений в условиях неопределенности,
необходимо рассмотреть классические модели, не использующие квантовый формализм. Особенность следующей модели
заключается в том, что ее модель схожа с квантовыми, но не рассматривает остальные социальные задачи, из-за чего
данная модель ограничена в применении.
Это модель социально-значимых Интернет ресурсов, разработанная на основе результатов исследований в социальной
психологии, в механизме запоминания информации человеком, а также на основе теории информации~\citep{pilkevich2015model}.
Данная модель в равной степени адекватно описывает поведение участников сети Интернет ресурсов.

Существует большое количество различных социальных Интернет-ресурсов.
И в первую очередь это различые социальные сети, благодаря которым человек получает персонифицированную информацию
как о себе, так и о других участниках.
Это блоги, благодаря которым человек получает более развернутую информацию.
Различные форумы, которые позволяют обсуждать некоторыую тему и др.
Для предоставления участникам ресурса возможности общения, во многих Интернет-ресурсах требуется предварительная
регистрация.
Определенного шаблона для составления анкеты на этапе регистрации не существует, поскольку различные Интерент-ресурсы
требуют не все данные о пользователе.
Однако, существует некоторая тенденция в составлении анкет во время регистрации, благодаря чему можно спроектировать
модель аккаунта пользователя.

В модели социально-значимых Интернет-ресурсов предлагается модель пользовательского аккаунта.
Эта модель описывает большое количество различных параметров анкеты пользователя.
Эта модель имеет следующий вид:
\begin{equation}
    A = <Pers, Cont, CoC, Prof>,
\end{equation}
где $Pers$ — подмножество идентифицирующей информации, $Cont$ — подмножество контактных данных,
$CoC$ — подмножество данных о социальных связях, $Prof$ — подмножество различной личной информации.

Помимо этого, для этой модели существует модель распространения информации, которая имеет вид:
\begin{equation}
    I = <Pers, M, T>,
\end{equation}
где $Pers$ — подмножество идентифицирующей информации, $M$ — мнения, комментарии,
$T$ — отметка времени, $K = f(M)$ — когниция, элемент знания (данные, усвоенные сознанием).

Помимо этого, модель пользовательского аккаунта имеет информацию о связях с другими аккаунтами,
которая позволяет рассчитать параметр интенсивности отношений:
\begin{equation}\label{alphaik}
    \alpha_{ik} = N log_{2} \frac{N_{ik}}{N_{0}},
\end{equation}
где $N_{ik}$ – количество переданых сообщений/знаков между членами $i$ и $k$,
$N_{0}$ – среднее значение переданных сообщений в сообществе, $N$ – некоторая константа, различная для каждого индивида.

На основе формулы \ref{alphaik}, а также модели пользовательского аккаунта и модели распространения информации
приводится схема взаимодействия пользователей и информации в виде графа (см. рисунок \ref{fig:shema_system_in_socium}).

\begin{figure}[h!]
    \centering
    \captionsetup{justification=centering}
    \includegraphics[width=0.8\linewidth]{pictures/shema_system_in_socium.png}
    \caption{Схема системы психологических отношений в социуме (социально значимых Интернет-ресурсах)}
    \label{fig:shema_system_in_socium}
\end{figure}

На этом рисунке $A = \{ A_{1}, A_{2}, A_{3}, \dots, A_{N}\}$ – это члены сети,
$K = \{ K_{1}, K_{2}, K_{3}, \dots, K_{N}\}$ – это когниции.~Схема построена с учетом теории когнитивного
~диссонанса Л.~Фестингера,~теории структурного баланса Ф.~Хайдера, теории коммуникационных актов Т.~Ньюкома
и теории конгруэнтности Ч.~Осгуда и П.~Танненбаума, для которых базовыми постулатами являются предположения
о возможности получения человеком информации, его переработке, его усвоению, о стремлении человека к наибольшей
связности имеющихся знаний, которые могут побуждать к действиям.

Самым~простым элементом отношений в этой схеме является рассмотрение двух членов социума (O,S) и одной когниции (K).
Смена значений этих отношений называется как информационно-психологическая акция (ИПА).
Данная ИПА представлена на Рисунке~\ref{fig:shema_OSK}.
\begin{figure}[h!]
    \centering
    \captionsetup{justification=centering}
    \includegraphics[width=0.7\linewidth]{pictures/schema_OSK.png}
    \caption{Схема отношений объекта O и субъекта S друг к другу и к утверждению K: a)
    для события до ИПА; б) после ИПА~\citep{pilkevich2015model}}
    \label{fig:shema_OSK}
\end{figure}
На рисунке~\ref{fig:shema_OSK}~а параметры $\alpha_{OS}, \alpha_{SO}$ показывают отношение объекта и субъекта
друг к другу, параметры $\beta_{OK}, \beta_{SK}$ показывают отношение членов социума к утверждению.
На рисунке~\ref{fig:shema_OSK}~б, эти отношения имеют новые значения и обозначаются
$\alpha^{\prime}_{OS}, \alpha^{\prime}_{SO}, \beta^{\prime}_{OK}, \beta^{\prime}_{SK}$
Изменение значений для параметров отношений вычисляются на основе дифференциалов Осгуда и для этой модели
записываются следующим образом:
\begin{equation}
    \Delta_{OK} = \frac{|\alpha_{OS}|}{|\beta_{OK}|+|\alpha_{OS}|} |\beta_{OK}-\alpha_{OS}|,\\
    \Delta_{OS} = \frac{|\beta_{OK}|}{|\beta_{OK}|+|\alpha_{OS}|} |\beta_{OK}-\alpha_{OS}|
\end{equation}
где $\Delta_{OK}, \Delta_{OS}$ – это приращение к $\beta_{OK}, \alpha_{OS}$ соответственно.
Различные ситуации взаимоотношений между элементами {O,S,K} были рассмотрены в теории структурного
баланса Ф.~Хайдера, а изменение отношений рассмотрено в теории коммуниуативных актов Т.~Ньюкомба,
благодаря чему известны формулы для $\beta^{\prime}_{OK}, \alpha^{\prime}_{OS}$, которые получаются
от начальных параметров отношений $\beta_{OK}, \alpha_{OS}$ и приращений к ним $\Delta_{OK}, \Delta_{OS}$.

В работах по моделированию информационно-психологических отношений наблюдается явление, при котором
элемент социума при получения некоторой когниции распространяет его вторично.
Формула, по которой это вторичное распространение информации описывется имеет вид:
\begin{equation}\label{beta_ij}
    \beta_{ij}(t) = \beta_{ij}(t_0)\cos(\rho_{i}\sqrt{t-t_0})e^{-\lambda_{i}(t-t_0)},
\end{equation}
где $t_0$ – время последнего изменения отношения элемента социума к когниции (утверждению), $t$ – текущее
время изменения, $\lambda_{i}$ – коэффициент «старения и утраты уникальности утверждения», $\rho_{i}$ –
коэффициент когнитивного «резонанса», вторичного распространения информации.
График данной функции представлен на рисунке~\ref{fig:graphic_beta}
\begin{figure}[h!]
    \captionsetup{justification=centering}
    \begin{minipage}[h]{0.4\linewidth}
        \center{\includegraphics[width=1\linewidth]{pictures/graphic_beta_a.png} \\ а)}
    \end{minipage}
    \hfill
    \begin{minipage}[h]{0.4\linewidth}
        \center{\includegraphics[width=1\linewidth]{pictures/graphic_beta_b.png} \\ б)}
    \end{minipage}
    \caption{Уровень отношения $\beta_{ij}(t)$ члена сети $A_{i}$ к утверждению $K_{j}$~\citep{pilkevich2015model}}
    \label{fig:graphic_beta}
\end{figure}

Данный график, к сожалению, не показывает других зависимостей этой функции от других параметров, но
на нем хорошо видно как ведет функция себя при постоянных параметрах.
На рисунке~\ref{fig:graphic_beta}а показан полный вариант с положитеьлными и отрицательными полупериодами.
Отрицательные полупериоды показывают маловероятное проявление когнитивной активности.
Положительные полупериоды показывает наиболее вероятное проявление когнитивной активности.
Для более детального рассмотрения графика функции $\beta_{ij}(t)$, на рисунке~\ref{fig:beta_another_param}
представлены зависимости с дополнительным параметром ($\lambda_{i}$).
\begin{figure}[h!]
    \centering
    \captionsetup{justification=centering}
    \includegraphics[width=0.7\linewidth]{pictures/graphic_beta_anotherparam.png}
    \caption{График функции $\beta_{ij}(t)$ с динамикой параметра $\lambda_{i}$~\citep{pilkevich2015model}}
    \label{fig:beta_another_param}
\end{figure}

Как видно из этого и предыдущего графиков, что существуют семейства монотонно убывающих зависимостей
Эббингауза при определенных значениях параметра $\lambda_{i}$.
Однако, как подчеркнули сами авторы, данную модель невозможно использовать в практической деятельности,
но предлагают ограничение этой модели в определенных коэффициентах для определенных однородных групп,
иначе говоря групп состоящих из людей с одним уровнем образования или предпочтений.
Таким образом диапазон модели для $4.5 < \rho_{i} < 5$ представлен на рисунке~\ref{fig:beta_4_5rho5}
\newpage
\begin{figure}[h!]
    \centering
    \captionsetup{justification=centering}
    \includegraphics[width=0.7\linewidth]{pictures/graphic_beta_45rho5.png}
    \caption{График функции $\beta_{ij}(t)$ с динамикой параметра $\rho_{i}$~\citep{pilkevich2015model}}
    \label{fig:beta_4_5rho5}
\end{figure}

В работе утверждается, что данное поведение похоже на социальную диффузию, которое представляется как
\glqqкогнитивный шум\grqq.
Интересно также то, что данный шум показывает момент перехода мнения человека к мнению группы, что так
же подтверждает справедливость уменьшения диапазонов $\rho_{i}$ и $\lambda_{i}$.

Теперь рассмотрим эту проблему с точки зрения известных квантово-подобных моделей.
%\section{Пояснение связей экспериментов и модели}
%Связь между моделью социално-значимых Интернет ресурсов и когнитивными экспериментами заключается в
%вывляении общих рассматриваемых объектов.
%Такими объектами в экспериментах является человек или два человека и некоторый вопрос или утверждение,
%а в модели имеются два человека и утверждение.
%Это значит, что множество объектов для экспериментов и для модели имеет одну и ту же структуру.
%Схема этой структуры представлена на рисунке~\ref{fig:chain_struct}
%\begin{figure}[h!]
%    \centering
%    \captionsetup{justification=centering}
%    \includegraphics[width=0.5\linewidth]{pictures/chain_struct.png}
%    \caption{Человек и утверждение}
%    \label{fig:chain_struct}
%\end{figure}
%
%В~данном~случае~человек (агент) находится внутри~этой когниции (утверждения), таким образом агент
%воспринимает когницию как окружающий мир.
%Взаимодействие между агентом и когницией подчинятеся модели социально-значимых ресурсов~\eqref{beta_ij}.
%Причина, по которой не указано два агента, заключается в отсутствии влияния другого агента на когницию.
%Влияние агента на когницию не происходит, а происходит изменение только отношения агента к когниции.
%Это легко заметить в окружающем мире, факты о каком-либо событии невозможно изменить, их можно только
%поменять или не указывать.

\section{Открытые квантовые системы}

Это такая система, которая может обмениваться информацией с внешней средой.
В некотором смысле, любая квантовая система является открытой, поскольку на нее могут влиять извне.
Эти воздействия извне создают квантовые шумы, которые мешают рассматривать квантовые системы в простом случае.
Открытые квантовые системы рассматривают ситуацию, когда имеется резервуар и система.
Система может иметь чистое состояние или смешанное и разница состоит в том, что чистое состояние можно
описать как векторами, так и матрицей плотности, а смешанное только матрицей плотности.
Соответственно, динамику для смешанного и для чистого состояния можно задать с помощью следующего уравнения:
\begin{equation}\label{heisenberg}
\frac{d \rho}{dt} = -i[H,\rho],
\end{equation}
где $H$ - гамильтониан, $\rho$ - матрица плотности.
Но уравнение~\eqref{heisenberg} не описывает динамику с учетом резервуара, поэтому появилось уравнение
Горини-Коссаковского-Сударшана-Линдблада (ГКСЛ):
\begin{equation}
    \frac{d \rho}{dt} = -i[H,\rho] +
    \sum_{j} (C_{j}\rho C^{\dagger}_{j} - \frac{1}{2} C^{\dagger}_{j} C_{j} \rho - \frac{1}{2} \rho C_{j} C^{\dagger}_{j}),
\end{equation}
где $C_{j}$ и $C^{\dagger}_{j}$ - некие операторы анигиляции и рождения, соответственно.
Это уравнение описывает открытые квантовые системы, которые наблюдаются в экспериментах.
Но это уравнение не единственное, которое может описывать открытые квантовые системы, поскольку
квантовые системы могут иметь разные квантовые состояния, которые зависят от резервуара.
А действие резервуара на систему может иметь марковский или немарковский характер.

%%%%%%%%%%%%%%%%%%%%%%%%%%%%%%%%%%%%%%%%%%%%%%%%%%%%%%%%%%%%%%%%%%%%%%%%%%%%%%%%%%%%%%%%%%%%%%%%%%%%
%%%%%%%%%%%%%%%%%%%%%%%%%%%%%%%%%%%%%%%%%%%%%%%%%%%%%%%%%%%%%%%%%%%%%%%%%%%%%%%%%%%%%%%%%%%%%%%%%%%%
\section{Марковские и немарковские системы}

Для рассмотрения немарковских процессов необходимо дать определение марковских процессов.
Марковские процессы — это, в упрощеном понимании, процесс не зависящий от предыдущих состояний системы,
который этот процесс реализует.
Иначе говоря, марковский процесс не имеет памяти и каждое новое состояние, если оно конечно, равновероятно
может произойти в следующий момент времени.
Самыми известными примерами марковских процессов является математическое описание броуновского движения
в газах или~случайное блуждание.
Броуновское движение наблюдается при перемещении микроскопических объектов~в~газах, а случайное блуждание
описывает~математическую~модель случайных изменений в траектории движения точки на плоскости.
Немарковские процессы — это процессы, в упрощенном понимании, зависящие от предыдущих состояний системы.
Самыми известными примерами такого поведения системы можно считать броуновское движение в жидкости,
когда при перемещении микрочастицы жидкость начинает передавать импульс другим частицам и создавать
завихрения, которые могут воздействовать и на частицу.
А так же фликер-шум, который наблюдается в электрических сетях.
Примеры немарковских процессов присутствуют так же и в социальных сетях, так например существует
модель социально значимых интернет-ресурсов.

Помимо этой социальных моделей имеются другие примеры немарковских открытых квантовых систем.
В одном из таких примеров рассматривалось немонотонное изменение параметра расстояния следа
$D(\rho^{1}_{S}, \rho^{1}_{S})$, в случае пары состояний $\rho^{1}_{S}$ и $\rho^{2}_{S}$.
Этот параметр характеризует различимость квантовых состояний.
Суть эксперимента состоит в том, что Алиса передает свои начальные состояния открытой квантовой
системы, связанной с окружающей средой, через зашумленный квантовый канал Бобу.
Из-за взаимодействия среды на канал, этот параметр начинает либо уменьшатся, либо увеличиваться со временем,
что означает либо потерю информации в окружающей среде, либо поступление информации из окружающей среды.
Данный пример схематически показан на рисунке~\ref{fig:state_from_alice_to_bob}.
\begin{figure}[h!]
    \centering
    \captionsetup{justification=centering}
    \includegraphics[width=0.7\linewidth]{pictures/state_alice_bob.png}
    \caption{Схема передачи состояния от Алисы к Бобу}
    \label{fig:state_from_alice_to_bob}
\end{figure}
В тот момент, когда система просто теряет информацию, имеет случай марковского процесса, а когда
система получает информацию, имеет случай немарковского процесса.
Получение информации связано с откликом окружающей среды на воздействие открытой квантовой системы,
что свидетельствует о наличии памяти у окружающей среды.

%%%%%%%%%%%%%%%%%%%%%%%%%%%%%%%%%%%%%%%%%%%%%%%%%%%%%%%%%%%%%%%%%%%%%%%%%%%%%%%%%%%%%%%%%%%%%%%%%%%%
%%%%%%%%%%%%%%%%%%%%%%%%%%%%%%%%%%%%%%%%%%%%%%%%%%%%%%%%%%%%%%%%%%%%%%%%%%%%%%%%%%%%%%%%%%%%%%%%%%%%
\section{Принятие решения в квантовом формализме}

%%%%%%%%%%%%%%%%%%%%%%%%%%%%%%%%%%%%%%%%%%%%%%%%%%%%%%%%%%%%%%%%%%%%%%%%%%%%%%%%%%%%%%%%%%%%%%%%%%%%
%%%%%%%%%%%%%%%%%%%%%%%%%%%%%%%%%%%%%%%%%%%%%%%%%%%%%%%%%%%%%%%%%%%%%%%%%%%%%%%%%%%%%%%%%%%%%%%%%%%%
\section{Дилемма заключенного в квантовом формализме}

Для дилеммы заключенного каждая квантовая модель рассматривается в гильбертовом пространстве, которое
может быть бесконечномерным.
Для системы из двух игроков достаточно конечного пространства.
Определение размерности зависит от количества исходов игры и для этого необходимо определить какие
есть варианты развития событий.
В задаче дилеммы заключенного участвуют двое игроков $G_{1}$ и $G_{2}$ , у каждого из которых есть по два возможных
варианта выбора это сдать и не сдавать, обозначим как вектора в гильбертовом пространстве
$\vert 0 \rangle$ и $\vert 1 \rangle$, соответственно.
Тогда, функцию состояния каждого игрока можно описать с помощью следующей формулы:
\begin{equation}
    \vert \psi \rangle_{j} = \sum_{j=0,1} a_{j} \vert j \rangle
\end{equation}
где $a_{j}$ - коэффициент проекции вектора $\vert j \rangle$.

Такую функцию можно представить с помощью двумерного гильбертова пространства $\mathbb{C}^{2}$.
Квадрат модуля коэффициента $\vert a_{j} \vert^{2}$ имеет смысл как вероятность выбора одного из
векторов $\vert 0 \rangle$ или $\vert 1 \rangle$.
В таком пространстве базисы векторов двух игроков могут быть повернуты на угол друг относительно друга,
а значит в таком случае при выборе одинакового значения двух игроков, направление вектора состояния
будет различаться из-за различного ментального восприятия мира.
При использовании данной формулы можно определить вариант выбора одного игрока, однако на основе
выбора первого невозможно определить выбор другого, иначе говоря их решение не влияет на результат
другого, что в данной дилемме не всегда так, поскольку при повторении игры прошлый опыт влияет на
обоих игроков одновременно.
Поэтому необходимо определить другую функцию состояния, которая впоследствии позволит правильно
определить размерность гильбертова пространства.
Поскольку выбор двух игроков должен влиять друг на друга, то получается они имеют общую функцию
состояния, имеющая следующий вид:
\begin{equation}
    \vert \Psi \rangle_{j} = \sum_{k,l=0,1}^{1} a_{k,l} \vert kl \rangle
\end{equation}
из этой формулы так же видно, что у игроков имеется общее ментальное состояние, что как раз более
правдоподобно в реальном мире.
По аналогии из формулы выше для двумерного гильбертова пространства, $a_{k,l}$ – коэффициент
проекции вектора $\vert kl \rangle$, где $\vert kl \rangle$ – вектор выбора двух игроков
(всего их четыре $\vert kl \rangle$ $\vert kl \rangle$ $\vert kl \rangle$ $\vert kl \rangle$),
а квадрат коэффициента проекции $\vert a_{j} \vert^{2}$ имеет смысл как вероятность выбора исхода игры.

%    % Глава 3
    %! Author = jaroslav
%! Date = 17.04.20
\chapter{КВАНТОВО-ПОДОБНАЯ МОДЕЛЬ~ПРИНЯТИЯ РЕШЕНИЙ В УСЛОВИЯХ НЕОПРЕДЕЛЕННОСТИ ВО ВЗАИМОДЕЙСТВИИ С ОКРУЖЕНИЕМ}
\section{Постановка задачи}

Ранее был рассмотрен эксперимент с дилеммой заключенного и его сформулированный эффект.
Этот эффект хорошо моделируется с помощью квантово-подобных моделей принятия решений, благодаря
тому, что неопределенность описывется квантовой суперпозицией с точки зрения одного из игроков.
Таким образом описание неопределенности для игрока А относительно игрока Б описывается следующим образом:
\begin{equation}\label{phi_B}
    \vert \phi_{B} \rangle = \alpha \vert 0_{B} \rangle + \beta \vert 1_{B} \rangle \in \mathbb{C}^{2} % \mathbf{C}^{2}
\end{equation}
где $\alpha$ и $\beta$ - показывают степень осознания действия игрока Б, $\vert 0_{B} \rangle$ - характеризует
вектор предательства игроком Б, $\vert 1_{B} \rangle$ - характеризует вектор сотрудничества игроком Б.
$\mathbb{C}^{2}$ - двумерное комплексное пространство (в данном случае Гильбертово).
Для описания ситуации с учетом принятия решения игрока А, необходимо использовать четырехмерное Гильбертово
пространство $\mathbb{C}^{2} \otimes \mathbb{C}^{2}$.
В таком случае уравнение \eqref{phi_B} имеет следующий вид, в случае предательства игрока А~\citep{asano2011quantum}:
\begin{equation}
    \vert \Phi_{0_{A}} \rangle = \alpha_{0} \vert 0_{A} 0_{B} \rangle + \beta_{0} \vert 0_{A} 1_{B} \rangle \in \mathbb{C}^{2} \otimes \mathbb{C}^{2}
\end{equation}
С учетом сотрудничества игрока А, вектор общего психического состояния системы двух игроков записывается
следующим образом:
\begin{equation}
    \vert \Psi \rangle = \sum_{x,y=0}^{1} \alpha_{x,y} \vert x_{A} y_{B} \rangle =
    \alpha_{0} \vert 0_{A} 0_{B} \rangle + \beta_{0} \vert 0_{A} 1_{B} \rangle +
    \alpha_{1} \vert 1_{A} 0_{B} \rangle + \beta_{1} \vert 1_{A} 1_{B} \rangle
\end{equation}

Этот подход для описания состояния двух игроков позволяет рассматривать проблему на коллективном уровне.
Так например можно применить теорию открытых квантовых систем и рассматривать игрока А как квантовую
систему, а игрока Б как резервуар~\citep{bagarello2015operator,bagarello2015quantum}.
В этом случае игрока А будем обозначать как Алиса или агент $a$, а игрока Б как резервуар $b$.
В квантовой теории поля применима процедура квантования, основанная на алгебре операторов рождения
и уничтожения $a, a^{\dagger}, b, b^{\dagger}$.
Эти операторы позволяют рассматривать переход из одного квантового состояния в другое.
\begin{equation}
    \vert 1_{A} 0_{B} \rangle = a^{\dagger} \vert 0_{A} 0_{B} \rangle,
    \vert 0_{A} 1_{B} \rangle = a b^{\dagger} \vert 1_{A} 0_{B} \rangle
\end{equation}
Для данных операторов действуют следующие коммутационные соотношения:
\begin{equation}\label{CCR}
    [ a, a^{\dagger} ] = 1,
    [ b(k), b^{\dagger}(k) ] = \delta (k-k')
\end{equation}
где $k$ - размернность резервуара, $[x,y] = xy - yx$ - коммутатор.
Зная коммутационное соотношение~\eqref{CCR}, можно определить числовой оператор $n_a=a^{\dagger}a$
для агента.
Собственное значение этого оператора соответствует выбору, которым оперирует агент~\citep{bagarello2015quantum}.
Для определения динамики операторов агента и резервуара используется уравнение Гейзенберга, которое имеет вид:
\begin{equation}
    i \hbar \frac{\partial f}{\partial t} = [f, H]
\end{equation}
где $i$ - мнимая единица, $\hbar$ - постоянная Планка, $\frac{\partial f}{\partial t}$ - дифференциал
функции $f$ по времени $t$, $[\cdot, \cdot]$ - коммутатор, $H$ - гамильтониан.
Полный гамильтониан является суммой гамильтониана агента, резервуара и их взаимодействия:
\begin{multline}
    H = H_f + H_R + H_{int} = \hbar \omega_a a^{\dagger} a + \hbar \int \omega_{R}(k) b^{\dagger}(k) b(k) d^{D}(k) \\
    + \hbar \lambda \int g(k)(b^{\dagger}(k) a + a^{\dagger} b(k)) d^{D}(k)
\end{multline}
где $H_f$ - гамильтониан описывающий агента, $H_R$ - гамильтониан описывающий резервуар, $H_{int}$ -
гамильтониан взаимодействия резервуара и агента, $\omega_{a}$ - частотная характеристика агента,
$\omega_{R}(k)$ - частотная характеристика резервуара, $g(k)$ - параметр плотности резервуара,
$\lambda$ - коэффициент взаимодействия резервуара и агента.
Операторы рождения(уничтожения) агента и резервуара в представлении Гейзенберга:
\begin{equation}
    \begin{cases}
        & \dot{a}(t) = -i \omega_a a(t) - i \lambda \int g(k)b(k,t) d^{D}(k) \\
        & \dot{b}(k,t) = -i \omega_{R}(k) b(k,t) -i \lambda g(k)a(t)
    \end{cases}
\end{equation}
Решение интегрально-операторного уравнения для динамики оператора $\dot{b}(k,t)$ с начальными
условиями $b(k,0)=b(k)$ имеет следующий вид:
\begin{equation}
    b(k,t) = b(k)e^{-i \omega_{R}(k) t}-i \lambda \int_{0}^{t} g(k)a(\tau)e^{-i \omega_{R}(k) (t-\tau)} d\tau
\end{equation}
В результате полученного решения для $b(k,t)$, можно найти уравнение для $a(t)$, который имеет следующий вид:
\begin{multline}\label{dotat_bagarello}
    \dot{a}(t) =
    -i \omega_{a} a(t)
    -i \lambda \int g(k)b(k)e^{-i \omega_{R}(k) t} d^{D}(k) \\
    -\lambda^{2} \iint_{0}^{t} g^{2}(k) a(\tau) e^{-i \omega_{R}(k) (t-\tau)} d\tau d^{D}(k)
\end{multline}
Решение этого уравнения является нетривиальной задачей, поскольку двойной интеграл в третьем слагаемом
никак не сокращается.
Частным решением этого уравнения является принятие постоянных значений переменных $g(k) = 1$ и
$\omega_{R}(k) = \omega_{b} k$~\citep{bagarello2018quantum}.
Благодаря чему уравнение~\eqref{dotat_bagarello} имеет следующий вид:
\begin{equation}
    \dot{a}(t) =
    - \Biggl(i \omega_{a} + \frac{\pi \lambda^{2}}{\omega_{b}} \Biggr) a(t)
    -i \lambda \int b(k)e^{-i \omega_{b} k t} d^{D}(k)
\end{equation}
Решение такого уравнения имеет простое аналитическое решение:
\begin{equation}\label{at_bagarello}
    a(t) = \Biggl( a - i \lambda \int b(k) \eta(k,t) d^{D}(k) \Biggr)
    e^{-\bigl(i \omega_{a} + \frac{\pi \lambda^{2}}{\omega_{b}} \bigr) t}
\end{equation}
где $\eta(k,t) = \frac{1}{\rho(k)}(e^{\rho(k) t} - 1)$,
$\rho(k) = i\omega - i\omega_{b} k + \frac{\pi \lambda^{2}}{\omega_{b}}$~\citep{bagarello2018quantum}.
Уравнение~\eqref{at_bagarello} подчиняется правилу коммутации.
Зная свойство для операторов резервуара $\langle b^{\dagger}(k) b^{k'} \rangle = n_{b} \delta (k-k')$,
а также свойство для динамики числового оператора $n_{a}(t) = a^{\dagger}(t) a(t)$, который в последующем
будем просто называть оператор принятия решения, можем получить решение числового оператора агента:
\begin{equation}\label{na_bagarello}
    n_{a}(t) = n_{a} e^{- \frac{2 \lambda \pi}{\omega_{b}} t} + n_{b} \Biggl( 1 - e^{\frac{2 \lambda \pi}{\omega_{b}} t} \Biggr)
\end{equation}
Эта модель при $ t \rightarrow \infty$ стремится к фиксированному значению $n_{b}$, что как раз
представлено на рисуке~\ref{fig:at_baga}
\begin{figure}[h!]
    \captionsetup{justification=centering}
    \begin{minipage}[h]{0.49\linewidth}
        \center{\includegraphics[width=1\linewidth]{pictures/at_baga_0.png} \\ а)} \\
    \end{minipage}
    \begin{minipage}[h]{0.49\linewidth}
        \center{\includegraphics[width=1\linewidth]{pictures/at_baga_01.png} \\ б)}
    \end{minipage}
    \begin{minipage}[h]{0.49\linewidth}
        \center{\includegraphics[width=1\linewidth]{pictures/at_baga_05.png} \\ в)} \\
    \end{minipage}
    \begin{minipage}[h]{0.49\linewidth}
        \center{\includegraphics[width=1\linewidth]{pictures/at_baga_1.png} \\ г)}
    \end{minipage}
    \caption{Характер поведения модели при $n_{b}$ равном: а)0; б)0,1; в)0,5; г)1}
    \label{fig:at_baga}
\end{figure}

Как видно из этого рисунка, характер поведения модели схож по поведению модели социально-значимых
Интернет ресурсов при параметре $\rho_{i} = 1$.
Однако, в данной модели не рассматриваются другие частные случаи для параметров $g(k)$ и $\omega_{R}(k)$.

%%%%%%%%%%%%%%%%%%%%%%%%%%%%%%%%%%%%%%%%%%%%%%%%%%%%%%%%%%%%%%%%%%%%%%%%%%%%%%%%%%%%%%%%%%%%%%%%%%%%
%%%%%%%%%%%%%%%%%%%%%%%%%%%%%%%%%%%%%%%%%%%%%%%%%%%%%%%%%%%%%%%%%%%%%%%%%%%%%%%%%%%%%%%%%%%%%%%%%%%%
\section{Основные уравнения}

В этой части рассматривается $\omega_{R}(k) = \omega k$, а параметр плотности $g(k)$ подчиняется
следующему уравнению:
\begin{equation}\label{gk}
    g^{2}(k) = \frac{g}{k^{2} + \alpha^{2}},
\end{equation}
Подставляя уравнение~\eqref{gk} в уравнение~\eqref{dotat_bagarello} и приняв во внимание тот факт, что
\begin{equation}
    \int \frac{e^{-i \omega k (t - \tau)}}{k^{2} + \alpha^{2}} dk = \frac{\pi}{\alpha} e^{-\alpha \omega (t - \tau)}
\end{equation}
получаем новый вид оператора $\dot{a}(t)$:
\begin{equation}\label{dotat}
    \dot{a}(t) =
    -i \omega_{a} a(t)
    -i \lambda \sqrt{g} \int \frac{b(k)e^{-i \omega k t}}{\sqrt{k^{2} + \alpha^2}} dk
    -\frac{\lambda^{2} g \pi}{\alpha} \int_{0}^{t} a(\tau) e^{- \alpha \omega (t - \tau)} d\tau
\end{equation}
Решать данное уравнение стандартно не получится, поскольку третье слагаемое невозможно вывести из
под знака интеграла.
В данной работе будет использоваться два способа решения этой задачи.
Первый способ будет заключаться в поиске непосредственно оператора принятия решения $n_{a}$ числовым
методом, не получая вид оператора агента $a^{\dagger}(a)$.
Второй способ заключается в поиске оператора рождения(уничтожения) для агента $a^{\dagger}(a)$ числовым
методом, а после числовым методом определить оператор принятия решения $n_{a}$.

Для первого способа, зная свойство оператора принятия решения $n_{a}(t) = a^{\dagger}(t) a(t)$, а также правило
дифференцирования $(xy)' = x'y + xy'$, можно получить итерационное числовое решение этой модели:
\begin{multline}
    n_{a}(t + \Delta t) =
    \Biggl[i \lambda \sqrt{g} \int \frac{R_{1}(k,t) e^{i \omega k t} - R_{2}(k,t) e^{-i \omega k t}}{\sqrt{k^{2} + \alpha^2}} dk - \\
    - 2 \frac{\lambda^{2} g \pi}{\alpha} n_{a}(t) \Biggr] \Delta t + n_{a}(t)
\end{multline}
\begin{multline}
    R_{1}(k,t + \Delta t) =
    \Biggl[-i \omega_{a} R_{1}(k,t)
    -i \lambda \sqrt{g} \frac{n_{b} e^{-i \omega k t}}{\sqrt{k^{2} + \alpha^2}} - \\
    -\frac{\lambda^{2} g \pi}{\alpha} \int_{0}^{t} R_{1}(k,\tau) e^{- \alpha \omega (t - \tau)} d\tau \Biggr] \Delta t + R_{1}(k,t)
\end{multline}
\begin{multline}
    R_{2}(k,t + \Delta t) =
    \Biggl[i \omega_{a} R_{2}(k,t)
    + i \lambda \sqrt{g} \frac{n_{b} e^{i \omega k t}}{\sqrt{k^{2} + \alpha^2}} + \\
    - \frac{\lambda^{2} g \pi}{\alpha} \int_{0}^{t} R_{2}(k,\tau) e^{- \alpha \omega (t - \tau)} d\tau \Biggr] \Delta t + R_{2}(k,t)
\end{multline}
Все стадии получения этих уравнений приведены в приложении.
Остается вопрос с параметрами для модели, какие они должны быть и подчиняются ли они какому-либо закону.
Правило по которому параметры не могут иметь случайные величины основано на коммутационном соотношении
для динамки операторов агента:
\begin{equation}\label{dynamic_comm-1}
    [\dot{a}(t),\dot{a}^{\dagger}(t')] = \frac{\partial}{\partial t} \frac{\partial}{\partial t'} \delta (t-t') =
    \dot{a}(t) \dot{a}^{\dagger}(t') - \dot{a}^{\dagger}(t') \dot{a}(t) = 0
\end{equation}
Это коммутационное соотношение имеет другой вид:
\begin{equation}\label{dynamic_comm}
    [\dot{a}(t),\dot{a}^{\dagger}(t')] = \omega^{2}_{a} + \lambda^{2} g \frac{\pi}{\alpha} +
    \frac{\lambda^{4} g^{2} \pi^{2}}{2 \alpha^{3} \omega} (1-e^{-2 \alpha \omega t}) = 0
\end{equation}
Все стадии вычислений находятся в приложении.
Это уравнение можно переписать в другом виде:
\begin{equation}
    g_{1,2} = \frac{\omega \alpha^{2}}{\lambda^{2} \pi (1-e^{-2 \alpha \omega t})}
    \Biggl( -1 \pm \sqrt{1 - 2 \frac{\omega^{2}_{a} }{\alpha \omega} (1-e^{-2 \alpha \omega t})} \Biggr)
\end{equation}
В дальнейшем разница между $g_{1}$ и $g_{2}$ не будет играть большой роли, поскольку это никак не будет
влиять на окончательный результат модели.

Стоит отметить, что $g \in \mathbb{R}$ и $g > 0$, связано это с тем, что параметр $g$ в уравнении~\eqref{dotat}
заключен под корнем и отрицательное значение $g$ приведет к комплексному значению параметра принятия решения.
Параметр принятия решения $n_a$ должен быть $n_a \in \mathbb{R}$.

%%%%%%%%%%%%%%%%%%%%%%%%%%%%%%%%%%%%%%%%%%%%%%%%%%%%%%%%%%%%%%%%%%%%%%%%%%%%%%%%%%%%%%%%%%%%%%%%%%%%
%%%%%%%%%%%%%%%%%%%%%%%%%%%%%%%%%%%%%%%%%%%%%%%%%%%%%%%%%%%%%%%%%%%%%%%%%%%%%%%%%%%%%%%%%%%%%%%%%%%%
\section{Методы решения}

Параметр $g$ имеет два решения:
\begin{equation}\label{gplus}
    g_{1} = \frac{\omega \alpha^{2}}{\lambda^{2} \pi (1-e^{-2 \alpha \omega t})}
    \Biggl( -1 + \sqrt{1 - 2 \frac{\omega^{2}_{a} }{\alpha \omega} (1-e^{-2 \alpha \omega t})} \Biggr),
\end{equation}
\begin{equation}\label{gminus}
    g_{2} = \frac{\omega \alpha^{2}}{\lambda^{2} \pi (1-e^{-2 \alpha \omega t})}
    \Biggl( -1 - \sqrt{1 - 2 \frac{\omega^{2}_{a} }{\alpha \omega} (1-e^{-2 \alpha \omega t})} \Biggr),
\end{equation}
Ранее для этого параметра было обнаружено, что его значение не должно быть комплексным или отрицательным,
поскольку этом случае для уравнения $n_{a}$ решения не существует.
Иначе говоря, оператор принятия решения $n_{a}$ становится комлексным числом.
Рассмотрим первое уравнение для параметра~\eqref{gplus}.
Это уравнение имеет решение, если $\alpha, \omega, \lambda, t \neq 0$.
Видно, что в начальный момент времени $t = 0$ это уравнение не применимо, однако этого не требуется
по решению основного уравнения, поскольку в начальный момент времени необходимые параметры уже заданы.
\begin{equation}\label{first_frac}
    \frac{\omega \alpha^{2}}{\lambda^{2} \pi (1-e^{-2 \alpha \omega t})},
\end{equation}
Первая дробь~\eqref{first_frac} имеет отрицательное значение при $\alpha < 0$.
В данном уравнении $\omega$ и $t$ не может быть отрицательным, поскольку время отсчитывается от
момента наблюдения, а отрицательной частоты быть не может.
Поэтому на знак вляет только два параметра.
Теперь рассмотрим в уравнении~\eqref{gplus} выражение под корнем:
\begin{equation}\label{last_square}
    \sqrt{1 - 2 \frac{\omega^{2}_{a} }{\alpha \omega} (1-e^{-2 \alpha \omega t})}
\end{equation}
Выражение~\eqref{last_square} должно быть строго положительное.
Теперь предположим, что $\alpha < 0, \omega > 0$.
В таком случае уравнение~\eqref{last_square} можем переписать в следующем виде:
\begin{equation}\label{last_square_two}
    \sqrt{1 + 2 \frac{\omega^{2}_{a} }{\vert \alpha \vert \omega} (1-e^{2 \vert \alpha \vert \omega t})}
\end{equation}
С~увеличением~времени~экспонента~начинает~возрастать,~а это значит, выражение в скобках будет~отрицательное~
$(1~-~e^{2 \vert \alpha \vert \omega t})~<~0$.
Раз оно отрицательное, значит второе слагаемое~под корнем будет так же отрицательное в выражении~\eqref{last_square_two}.
Поскольку все переменные в выражениях~\eqref{last_square} и~\eqref{last_square_two} являются
действительными числами, то на знак под корнем может зависеть только от $\alpha$.
Значит, второе слагаемое в выражениях~\eqref{last_square} и~\eqref{last_square_two} всегда будет отрицательное.

В таком случае рассмотрим следующее неравенство:
\begin{equation}
    2 \frac{\omega^{2}_{a} }{\alpha \omega} (1-e^{-2 \alpha \omega t}) < 1
\end{equation}
Если рассмотреть предел по $ t \rightarrow \infty$, то левая часть неравенства стремится к бесконечности.
Изменением переменных $\alpha$ и $\omega$, неравенство не соблюдается, покольку при увеличении этих
переменных, увеличивается и степень в экспоненте.
Значит необходимо определить границы переменной $\omega_{a}$, которые приведены в выражении~\eqref{omega_a}:
\begin{equation}\label{omega_a}
    \omega^{2}_{a} < \frac{\alpha \omega}{2(1-e^{2 \alpha \omega t})}
\end{equation}
Рассмотрим правую часть неравенства в пределе $t \rightarrow \infty$.
Пусть в этом случае $\alpha > 0$, тогда в пределе неравенство имеет следующий вид:
\begin{equation}\label{omega_a_2}
    \omega^{2}_{a} < \frac{\alpha \omega}{2}
\end{equation}
в случае $\alpha < 0$:
\begin{equation}\label{omega_a_3}
    \omega^{2}_{a} < \frac{\alpha \omega}{\infty} \rightarrow \omega^{2}_{a} = 0
\end{equation}
причина, по которой неравенство превратилось в равенство для $\omega_{a}$, заключается в неосуществимости
отричательной частоты.
Случай выражения~\eqref{omega_a_2} легко соблюдать, однако если решать уравнение~\eqref{gplus} с
параметром $\alpha > 0$, то зачение выражения~\eqref{gplus} будет $g < 0$, что не допустимо.
Значит нужно рассматривать случай для выражения~\eqref{omega_a_3}, поскольку только в этом случае $g > 0$.
Такие же значения выражения~\eqref{gplus} при данных параметрах, справедливы для выражения~\eqref{gminus}.
Из этого заключаем вывод, что значения должны быть следующие: $\omega^{2}_{a} = 0, \alpha < 0$.

Если внимательно рассмотреть результаты ограничений, то можно заметить отсутствие частотной характеристики
агента $\omega^{2}_{a} = 0$, чего не может происходить в рассматриваемой модели, поскольку это означает
отсутвие самого агента в резервуаре.
Отсутвие агента в резервуаре делает бессмысленным дальнейшее моделирование.
Было выдвинуто предположение, что была совершена ошибка во время вычисления и для выявления ошибки
необходимо проверить величины для каждого параметра: $k$ - безразмерный, $\lambda$ - безразмерный,
$g(k)$ - [Гц], $\omega$ - [Гц], $\omega_{a}$ - [Гц], $\alpha$ - безразмерный.
В результате было обнаружено, что в коммутаторе~\eqref{dynamic_comm} не совпадают размерности при вычислении.
Для чего была сделана нормировка значений по $g$ начиная с уравнений гамильтонианов, результаты которых
приведены в приложении.
Однако такой способ исключения ошибки не дал результатов в уравнении~\eqref{dynamic_comm}, поэтому
вопрос об ошибке в вычислениях остается открытым.
Возможно ошибка случается из-за того, что дельта-функция Дирака является не дифференцируемой функцией,
а значит и коммутатор~\eqref{dynamic_comm-1} нельзя приравнять к нулю.

%%%%%%%%%%%%%%%%%%%%%%%%%%%%%%%%%%%%%%%%%%%%%%%%%%%%%%%%%%%%%%%%%%%%%%%%%%%%%%%%%%%%%%%%%%%%%%%%%%%%
%%%%%%%%%%%%%%%%%%%%%%%%%%%%%%%%%%%%%%%%%%%%%%%%%%%%%%%%%%%%%%%%%%%%%%%%%%%%%%%%%%%%%%%%%%%%%%%%%%%%
%\section{Результаты вычисления первым способом}

Даже несмотря на присутствие комплексной составляющей в результатах вычислений, получить результат
схожий с марковскими квантово-подобными моделями оказалось невозможным для первого способа.
Также, не удалось получить подобие модели~социально-значимых Интернет ресурсов, но результаты и их
объяснение стоит привести для понимания проблемы моделирования.
Прежде всего стоит начать с тех результатов полученной модели, при котором наблюдается осцилляции в
поведении модели.
Разработанная модель подразумевает осцилляции при любых параметрах, и таким образом они будут присутствовать
всегда, разница будет заключаться в том насколько сильно они будут влиять на конечный результат.
Влияние на конечный результат может наблюдаться совершенно по разному и при порой неожиданных результатах.
У модели, разработанной данным способом, очень сложно предсказывается поведение как раз по причине
дополнительной комплексной составляющей, который дает параметр $g$.
Так например на рисунке~\ref{fig:fr_oscillation} осилляции очень быстро выводят агента из состояния
равновесия, причем характер осцилляций даже усиливается.
\begin{figure}[h!]
    \centering
    \captionsetup{justification=centering}
    \includegraphics[width=0.65\linewidth]{pictures/result_first_1.png}
    \caption{Осциллирующее поведение}
    \label{fig:fr_oscillation}
\end{figure}

Такой результат сложно~объяснить с точки~зрения когнитивной психологии, поскольку экспериментов с таким
эффектом найти не удалось.
Результат моделирования приведенный на рисунке~\ref{fig:fr_oscillation} можно описать как "метание"
агента из одного выбора к другому, когда человеку через некоторое время кажется, что он сделал фатальную
ошибку в выборе, но такое объяснение, к сожалению, не удалось привести из-за отсутствия реальных
экспериментов.
Помимо необъяснимости полученного результата, приведенного на рисунке~\ref{fig:fr_oscillation}, стоит
отметить отсутствие комплексной плоскости на графике, поскольку присутствие комплексной составляющей
в результате моделирования дает предпосылки к ошибке вычисления, а это необходимо учитывать при моделировании.
Такой же результат без комплексной плоскости необходимо отметить на следующем графике (рисунок~\ref{fig:fr_next}).
\begin{figure}[h!]
    \centering
    \captionsetup{justification=centering}
    \includegraphics[width=0.7\linewidth]{pictures/result_first_2.png}
    \caption{Экспоненциальное затухание когнитивного возбуждения с пичком}
    \label{fig:fr_next}
\end{figure}

На~рисунке~\ref{fig:fr_next}~происходит~некоторое~временное возбуждение с последующим затуханием,~поведение
которого~также объясняется комплексным или отрицательным параметром плотности $g$.
В когнитивистике это поведение можно объяснить как ситуацию с отрицанием противоположной точки зрения.
Когда среда начинает воздействовать на сознание противоречивой информацией, вызывая у человека когнитивный
диссонанс, проявляющаяся ответной реакцией в виде пичка на графике, но через некоторое время ответная
реакция сознания индивида постепенно затухает, иначе говоря индивид переходит к состноянию среды.

Более правильным поведением является случай с параметром плотности $g = 1$, по причине отсутствия
комплексной составляющей в результатах модели, иначе говоря $n_{a} \in \mathbb{R}$.
Конечно, данное действие не является правильным с точки зрения вычислений, поскольку в таком случае
коммутатор в уравнении~\eqref{dynamic_comm} не равняется нулю, а это значит что оператор рождения и
уничтожения агента не коммутируют, чего не должно происходить.
Поведение при $g = 1$ показано на рисунке~\ref{fig:fr_real_g_15}.
\begin{figure}[h!]
    \centering
    \captionsetup{justification=centering}
    \includegraphics[width=0.7\linewidth]{pictures/result_first_4.png}
    \caption{Экспоненциальное затухание при вещественном значении параметра $g$}
    \label{fig:fr_real_g_15}
\end{figure}
Как видно из рисунка~\ref{fig:fr_real_g_15} его поведение схоже с марковской квантово-подобной моделью,
про которую было сказано ранее, приведенной в уравнении~\eqref{na_bagarello} и на рисунке~\ref{fig:at_baga}.
С точки зрения квантовой когнитивистики это поведение было описано ранее в других
работах~\citep{bagarello2015quantum,bagarello2018quantum}.

В залючении этой части необходимо уточнить причины проблемы присутствия комплексной составляющей в модели.
Проблема заключается не только в том, что параметр плотности $g$ вносит комплексную составляющую в оператор
принятия решения $n_{a} \in \mathbb{C}$, но и в том, что поскольку даже отнормировав основные уравнения
по $g$, из-за чего в коммутаторе~\eqref{dynamic_comm} он исчезает, отрицаельными или комплексными окажутся
другие параметры, так как необходимо соблюдать коммутационные соотношения.
Решением этой проблемы может быть поиск аналитического способа, но в данной работе этого сделать не удалось.
В таком случае необходимо найти приближенный вид аналитического решения, что может быть осуществимо.

%%%%%%%%%%%%%%%%%%%%%%%%%%%%%%%%%%%%%%%%%%%%%%%%%%%%%%%%%%%%%%%%%%%%%%%%%%%%%%%%%%%%%%%%%%%%%%%%%%%%
%%%%%%%%%%%%%%%%%%%%%%%%%%%%%%%%%%%%%%%%%%%%%%%%%%%%%%%%%%%%%%%%%%%%%%%%%%%%%%%%%%%%%%%%%%%%%%%%%%%%
%\section{Второй способ моделирования}
\section{Моделирование резервуара}

Для приближенного аналитического решения необходимо найти численное решение уравнения для оператора
агента $a(t), a^{\dagger}(t)$.
В основу численного решения было взято нормированное по $g$ уравнение~\eqref{dotat_norm}, вывод которого
подробно приведен в приложении.
\begin{equation}\label{dotat_norm}
    \dot{a}(t) =
    -i \omega^{'}_{a} a(t)
    -i \lambda \int \frac{b(k)e^{-i \omega k t}}{\sqrt{k^{2} + \alpha^2}} dk
    -\frac{\lambda^{2} \pi}{\alpha} \int_{0}^{t} a(\tau) e^{- \alpha \omega (t - \tau)} d\tau
\end{equation}
Как видно из уравнения~\eqref{dotat_norm} в нем отсутствует параметр плотности, а это значит его
задавать не нужно.
Применяем итерационный метод решения для уравнения~\eqref{dotat_norm}, который имеет вид:
\begin{multline}
    a(t + \Delta t) =
    -i \omega^{'}_{a} a(t) \Delta t
    -i \lambda \int \frac{b(k)e^{-i \omega k t}}{\sqrt{k^{2} + \alpha^2}} dk \Delta t \\
    -\frac{\lambda^{2} \pi}{\alpha} \int_{0}^{t} a(\tau) e^{- \alpha \omega (t - \tau)} d\tau \Delta t
    - a(t)
\end{multline}
Все параметры в этом уравнении задаются случайным способом в диапазоне вещественных чисел кроме одного.
Этим параметром является оператор резервуара $b(k), b^{\dagger}(k)$, который задается в области
комплексных значений.
В ходе расчетов было выяснено, что большую зависимость оператор агента $a(t), a^{\dagger}(t)$ проявляет
в зависимости от оператора резервуара $b(k), b^{\dagger}(k)$, а это зачит необходимо продумать
приближенные значения этого параметра в итерационном методе решения.
Поскольку известно, что операторы рождения и уничтожения комплексно-сопряженные между собой, то значит
можно предположить отличия между оператором рождения и уничтожения только знаком комплексной составляющей.
Помимо этого, параметр $b(k), b^{\dagger}(k)$ не просто число, а некоторая зависимость от параметра $k$.
Такая зависимость может иметь большое количество вариантов, из которых выявить основные не получится.
Моделирование оператора рождения и уничтожения резервуара $b(k), b^{\dagger}(k)$ нетривиальная задача,
которая требует отдельного рассмотрения, но без него невозможны дальнейшие действия.
В таком случае оператор рождения и уничтожения для резервуара были приняты равными константе
$b(k) = b, b^{\dagger}(k) = b^{\dagger}$, при этом они должны быть комплексно-сопряженными.

После численного решения оператора рождения и уничтожения агента $a(t), a^{\dagger}(t)$ остается
вопрос о проверке результатов.
Проверка результатов может быть проведена с помощью коммутационных соотношений $[a(t), a^{\dagger}(t)] = 1$,
которые уже были показаны в данной работе.
Эти коммутационные соотношения работают только для $a(t), a^{\dagger}(t)$, но не для
$\dot{a}(t), \dot{a}^{\dagger}(t)$, поэтому нужно найти хотя бы приближенное аналитическое решение.
Хорошим показателем правильного моделирования является отсутствие комплексного составляющего в результате
получения оператора принятия решения $n_{a}$, но для данного метода решения этого можно не делать,
поскольку параметры задаются вещественными в начале моделирования.
Получив итерационным методом оператор рождения и уничтожения агента $a(t), a^{\dagger}(t)$, можно
перейти к численному результату оператора принятия решения $n_{a}$.
Результаты численного решения оператора принятия решения преведны в следующей части главы.

%%%%%%%%%%%%%%%%%%%%%%%%%%%%%%%%%%%%%%%%%%%%%%%%%%%%%%%%%%%%%%%%%%%%%%%%%%%%%%%%%%%%%%%%%%%%%%%%%%%%
%%%%%%%%%%%%%%%%%%%%%%%%%%%%%%%%%%%%%%%%%%%%%%%%%%%%%%%%%%%%%%%%%%%%%%%%%%%%%%%%%%%%%%%%%%%%%%%%%%%%
%\section{Результаты вычисления вторым способом}

Для данного способа решения полученные данные оказались более правильными, поскольку комплексная составляющая
отсутствовала для всех результатов моделирования.
Результаты моделирования хорошо показывают проявление немарковского процесса, эффекта влияния "памяти"
на принятие решения, постепенный переход к равновесию в информационном резервуаре.
Разработанная модель очень просто сводится к результатам марковской квантово-подобной модели~\citep{bagarello2018quantum},
если задать степень влияния осцилляций на модель достаточно слабыми, а параметр $\alpha$ задать гораздо
больше чем в два раза частотной характеристики $\omega_{a}$, что как раз приведено на рисунке~\ref{fig:sr_mark}.
\begin{figure}[h!]
    \centering
    \captionsetup{justification=centering}
    \includegraphics[width=0.7\linewidth]{pictures/result_second_1.png}
    \caption{Марковское приближение при малых осцилляциях резервуара и $\alpha = 3$}
    \label{fig:sr_mark}
\end{figure}

Для сравнения приведенного результата на рисунке~\ref{fig:sr_mark} с марковской квантово-подобной
моделью принятия решения можно обратиться к рисунку~\ref{fig:at_baga}.
Этот результат не представляет интереса в даннной работе, поскольку такие результаты обсуждались в других работах.
Другой случай показывает влияние немарковского процесса на поведение агента при воздействии резервуара,
который приведен на рисунке~\ref{fig:sr_nomark}.
\begin{figure}[h!]
    \centering
    \captionsetup{justification=centering}
    \includegraphics[width=0.7\linewidth]{pictures/result_second_2.png}
    \caption{Немарковское поведение при малых осцилляциях резервуара и $\alpha = 0.8$}
    \label{fig:sr_nomark}
\end{figure}

Как видно из рисунка~\ref{fig:sr_nomark} агент стремится к начальному решению через некоторое время
равному $t = 100, 200, 300$, происходят осциляции с затуханием.
Такое поведение, с точки зрения конитивистики, объясняется как "когнитивный шум" и наблюдается в
социальных сетях при воздействии информационной среды сообщества на участника этого сообщества.
Как видно из рисунка~\ref{fig:sr_mark} и рисунка~\ref{fig:sr_nomark} немарковский процесс может
переходить в марковский, в случае малых величин оператора рождения(уничтожения) резервуара $b, b^{\dagger}$
и параметра $\alpha$ больше частотной характеристики агента $\omega_{a}$.
Было обнаружено, что при уменьшении параметра $\alpha$ относительно частотной характеристики $\omega_{a}$,
увеличивается частота осциляций модели, что в данном случае наблюдается на рисунке~\ref{fig:sr_more_oscillation}.
\newpage
\begin{figure}[h!]
    \centering
    \captionsetup{justification=centering}
    \includegraphics[width=0.7\linewidth]{pictures/result_second_3.png}
    \caption{Немарковское поведение при малых осцилляциях резервуара и $\alpha = 0.4$}
    \label{fig:sr_more_oscillation}
\end{figure}
Необходимо отметить зависимость диапазона моделирования от величины модуля оператора рождения(уничтожения)
резевуара $\arrowvert b \arrowvert, \arrowvert b^{\dagger} \arrowvert$.
Поскольку диапазон принятия решения находится в пределах от 0 до 1, то и параметры модели должны быть
подобраны таким образом, чтобы в результате соответсвовать этим ограничениям.
Поэтому сумма модулей оператора рождения и уничтожения резервуара не должны превышать 1, иначе говоря
сумма радиусов их векторов на комплексной плоскости не должны первышать 1.
На рисунке~\ref{fig:sr_more_oscillation} можно заметить, что нижние полупериоды осцилляций модели достигают
минимума постепенно, по закону гауссовой функции.
Это поведение можно наблюдать в физике при изучении степени пространственной когерентности реальных
лазерных источников в интерверометре Юнга, когда темные полосы интерференционной картины "засвечены"
нескоррелированными фотонами.
Такое поведение моделируется если сумма действительной части оператора рождения и уничтожения резервуара
близка по сумме к 1.
Более наглядное представление этого поведения представлено на рисунке~\ref{fig:sr_gauss}.
\begin{figure}[h!]
    \centering
    \captionsetup{justification=centering}
    \includegraphics[width=0.7\linewidth]{pictures/result_second_4.png}
    \caption{Немарковское поведение с возвышенными нижними полупериодами осцилляций}
    \label{fig:sr_gauss}
\end{figure}
Песледний вопрос в этой части касается приближенного аналитического вида модели принятия решения.
Для решения вопроса необходимо описать основные уравнения, которые будут составлять характер модели.
Первое что необходимо отметить, это присутствие осцилляций, характер которых может описать уравнение
косинуса.
Поскольку косинус находится в пределе от -1 до 1, то его значения нужно увеличить на единицу.
Осцилляции в модели со временем затухают и их характер затухания необходимо подбирать.
Для уравнения затухания было выбрано уравнение экспоненты в отрицательной степени, которая составляет
проезведение с уравнением косинуса.
В вышеприведенных результатах поведение агента стремится к покою(к 0) и это значит присутствие экспоненты
в знаменателе для всего полученного аналитического решения.
Окончательный результат этой приближенной модели представлен следующим образом:
\begin{equation}
    f(t) = \frac{cos(tx - y)*e^{-tz} + 1}{e^{t}},
\end{equation}
где $x$ - параметр частоты, $y$ - параметр сдвига, $z$ - параметр затухания осцилляций, $t$ - время.
Для демонстрации результатов поиска приближенного решения оператора принятия решения следует привести
уравнения с подобраными параметрами.
Первое уравнение имеет следующий вид:
\begin{equation}
    f(t) = \frac{cos(10*t/1.645 - 10)*e^{-t*0.3} + 1}{e^{t}},
\end{equation}
и его результат с приведенным немарковским поведением модели при сильных осциллциях и их "сильных" минимумов~\ref{fig:sr_proba_1}.
\begin{figure}[h!]
    \centering
    \captionsetup{justification=centering}
    \includegraphics[width=0.7\linewidth]{pictures/result_second_5.png}
    \caption{Поведение модели и приблеженного аналитического решения}
    \label{fig:sr_proba_1}
\end{figure}
Второе уравнение имеет следующий вид:
\begin{equation}
    f(t) = \frac{cos(10*t/1.645 - 9.5)*e^{-t*0.8} + 1}{e^{t}},
\end{equation}
где его результат с вышеприведенным поведением приведен на следующем рисунке~\ref{fig:sr_proba_2}.
\newpage
\begin{figure}[h!]
    \centering
    \captionsetup{justification=centering}
    \includegraphics[width=0.7\linewidth]{pictures/result_second_6.png}
    \caption{Поведение модели и приблеженного аналитического решения с другими параметрами}
    \label{fig:sr_proba_2}
\end{figure}

Коммутационное соотношение из приближенного уравнения взять не получится, поскольку отсутствует
оператор рождения(уничтожения) в слагаемых.
Но основная задача данной работы решена, поскольку была получена модель, описывающая эмпирические
данные в социальных сетях.

    % Заключение
    %! Author = jaroslav
%! Date = 22.04.20
\conclusion

В~ходе~работы были изучены известные когнитивные эффекты происходящие в условиях неопределенности.
Присутствие неопределенности вносит искажения в мышление человека и приводит к нелогическим выводам.
Нелогические выводы скорее всего связаны с невозможностью удержать в мыслях все возможные исходы,
что как раз предполагают в своих выводах Канеман и Тверски~\citep{tversky1983extensional}.
Невозможность удержать все факты в уме подтверждается экспериментами Эббингауза, когда человек
постепенно забывал только что поступившую к нему информацию, а в некоторых случаях даже искажал~\citep{bartlett2002human}.
В экспериментах Эббингауза кривая забывания описывается такой же кривой как в модели
социально-значимых Интернет ресурсов, что подтверждает вляние памяти на отношение к некоторому утверждению.
Раз отношение человека к некоторому утверждению изменяется со временем из-за влияния эффектов памяти,
значит и при принятии решения у человека может меняться выбор.
В случае дилеммы заключенного варианты выбора являются противоречивыми, однако человек рассматривает
их единым целым, поскольку вопрос был задан один для любого варианта, из-за чего факты в пользу одного
варианта или другого могут накладываться друг на друга из-за эффекта памяти и таким образом запутываться.
Явление запутанности давно изучено в квантовой физике, благодаря квантовым эффектам наблюдаемым
в окружающем мире, но помимо этого стоит отметить хорошо развитый математический формализм,
благодаря которому можно описать эффекты запутывания.

    % Список сокращений и условных обозначений
    \defs
    \begin{description}
        \item[ИПА] - информационно-психологическая акция
        \item[ПР] - принятие решения
        \item[ООП] - объектно-ориентированное программирование
    \end{description}

    % Не добавлять длинное тире в качестве разделителя
%    \newcommand\BibDash{}
    % Выделять курсивом
%    \let\BibEmph=\emph
%    \bibliographystyle{unsrt}
%    \bibliographystyle{ugost2008n}
    \bibliographystyle{ugost2003}

    % Список литературы
    \bibliography{sources}

    % Приложения
    \appendix
    %! Author = jaroslav
%! Date = 03.03.20
\chapter{Основные математические модели }

Задан гамильтониан эволюции:
\begin{equation}\label{hf}
    H_f = \hbar \omega_a a^{\dagger} a,
\end{equation}
Задан гамильтониан, описывающий резервуар:
\begin{equation}\label{hr}
    H_R = \hbar \int \omega_{R}(k) b^{\dagger}(k) b(k) d^{D}(k),
\end{equation}
Задан гамильтониан взимодействия резервуара и моды:
\begin{equation}\label{hint}
    H_{int} = \hbar \lambda \int g(k) (b^{\dagger}(k) a + a^{\dagger} b(k)) d^{D}(k),
\end{equation}
Для данной системы выполняются следующие коммутационные соотношения:
$
[\hat{a}, \hat{a}^{\dagger}] = 1,
[\hat{b}(k), \hat{b}^{\dagger}(k)] = \delta (k-k')
$
Приняв для данной системы ядро вида:
\begin{equation}
    g^{2}(k) = \frac{g}{k^{2} + \alpha^{2}},
\end{equation}
и отнормировав по $g$ для уравнений (\ref{hf}-\ref{hint}), получаем следующие выражения:
\begin{gather}
    H_f = \hbar \omega^{'}_a a^{\dagger} a, \\
    H_R = \hbar \int \omega^{'}_{R}(k) b^{\dagger}(k) b(k) d^{D}(k), \\
    H_{int} = \hbar \lambda \int \frac{(b^{\dagger}(k) a + a^{\dagger} b(k))}{\sqrt{k^{2} + \alpha^{2}}} d^{D}(k),
\end{gather}
где $\omega^{'}_a = \omega_a/g$, $\omega^{'}_{R}(k) = \omega_{R}(k)/g$

Полный гамильтониан системы состоит из суммы этих уравнений:
\begin{multline}
    H = H_f + H_R + H_{int} = \hbar \omega^{'}_a a^{\dagger} a + \hbar \int \omega^{'}_{R}(k) b^{\dagger}(k) b(k) d^{D}(k) \\
    + \hbar \lambda \int \frac{(b^{\dagger}(k) a + a^{\dagger} b(k))}{\sqrt{k^{2} + \alpha^{2}}} d^{D}(k)
\end{multline}
Зная коммутационные соотношения и полный гамильтониан системы можно решить уравнение Гейзенберга
для операторов $a$ и $b(k)$. Для оператора $b(k)$ будет следующее решение:
\begin{equation}\notag
    \begin{split}
        i \hbar \frac{\partial b(k,t)}{\partial t} & = [b(k,t), H] = \\
        & = \hbar \omega^{'}_a b(k,t) a^{\dagger}(t) a(t) + \hbar \int \omega^{'}_{R}(k) b(k,t) b^{\dagger}(k,t) b(k,t) d^{D}(k) + \\
        & + \hbar \lambda \int \frac{(b(k,t) b^{\dagger}(k,t) a(t) + b(k,t) a^{\dagger}(t) b(k,t))}{\sqrt{k^{2} + \alpha^{2}}} d^{D}(k) - \\
        & - \hbar \omega^{'}_a a^{\dagger}(t) a(t) b(k,t) - \hbar \int \omega^{'}_{R}(k) b^{\dagger}(k,t) b(k,t) b(k,t) d^{D}(k) - \\
        & - \hbar \lambda \int \frac{(b^{\dagger}(k,t) a(t) b(k,t) + a^{\dagger}(t) b(k,t) b(k,t))}{\sqrt{k^{2} + \alpha^{2}}} d^{D}(k) = \\
        & = xx' + yy' + zz' = \hbar \omega^{'}_{R}(k) b(k,t) + \hbar \lambda \frac{a(t)}{\sqrt{k^{2} + \alpha^{2}}}
    \end{split}
\end{equation}
где
\begin{equation}\notag
    \begin{split}
        xx' & = \hbar \omega^{'}_a b(k,t) a^{\dagger}(t) a(t) - \hbar \omega^{'}_a a^{\dagger}(t) a(t) b(k,t) = \\
        & = \cancel{\hbar \omega^{'}_a a^{\dagger}(t) b(k,t) a(t)} - \hbar \omega^{'}_a \underbrace{[a^{\dagger}(t), b(k,t)]}_{0} a(t) - \\
        & - \hbar \omega^{'}_a a^{\dagger}(t) \underbrace{[a(t), b(k,t)]}_{0} - \cancel{\hbar \omega^{'}_a a^{\dagger}(t) b(k,t) a(t)} = \\
        & = 0
    \end{split}
\end{equation}
\begin{equation}\notag
    \begin{split}
        yy' & = \hbar \int \omega^{'}_{R}(k) b(k,t) b^{\dagger}(k,t) b(k,t) d^{D}(k) - \\
        & - \hbar \int \omega^{'}_{R}(k) b^{\dagger}(k,t) b(k,t) b(k,t) d^{D}(k) = \\
        & = \hbar \int \omega^{'}_{R}(k) \Bigl([b(k,t), b^{\dagger}(k,t)] b(k,t) + \\
        & + \cancel{b^{\dagger}(k,t) b(k,t) b(k,t)} - \cancel{b^{\dagger}(k,t) b(k,t) b(k,t)} \Bigr) d^{D}(k) = \\
        & = \hbar \int \omega^{'}_{R}(k) \delta (k-k') b(k,t) d^{D}(k) = \hbar \omega^{'}_{R}(k') b(k',t)
    \end{split}
\end{equation}
\begin{equation}\notag
    \begin{split}
        zz' & = \hbar \lambda \int \frac{(b(k,t) b^{\dagger}(k,t) a(t) + b(k,t) a^{\dagger}(t) b(k,t))}{\sqrt{k^{2} + \alpha^{2}}} d^{D}(k) - \\
        & - \hbar \lambda \int \frac{(b^{\dagger}(k,t) a(t) b(k,t) + a^{\dagger}(t) b(k,t) b(k,t))}{\sqrt{k^{2} + \alpha^{2}}} d^{D}(k) = \\
        & = \hbar \lambda \int \frac{[b(k,t), b^{\dagger}(k,t)] a(t) + \cancel{b^{\dagger}(k,t) b(k,t) a(t)}}{\sqrt{k^{2} + \alpha^{2}}} d^{D}(k) - \\
        & - \hbar \lambda \int \frac{\overbrace{[a^{\dagger}(t), b(k,t)]}^{0} b(k,t) + \cancel{a^{\dagger}(t) b(k,t) b(k,t)}}{\sqrt{k^{2} + \alpha^{2}}} d^{D}(k) - \\
        & - \hbar \lambda \int \frac{b^{\dagger}(k,t) \overbrace{[a(t), b(k,t)]}^{0} + \cancel{b^{\dagger}(k,t) b(k,t) a(t)} + }{\sqrt{k^{2} + \alpha^{2}}} d^{D}(k) = \\
        & + \hbar \lambda \int \frac{\cancel{a^{\dagger}(t) b(k,t) b(k,t)}}{\sqrt{k^{2} + \alpha^{2}}} d^{D}(k) = \\
        & = \hbar \lambda \int \frac{[b(k,t), b^{\dagger}(k,t)] a(t)}{\sqrt{k^{2} + \alpha^{2}}} d^{D}(k) = \hbar \lambda \int \frac{\delta (k-k') a(t)}{\sqrt{k^{2} + \alpha^{2}}} d^{D}(k) = \\
        & = \hbar \lambda \frac{a(t)}{\sqrt{k^{2} + \alpha^{2}}}
    \end{split}
\end{equation}

Для оператора $a$ решение:
\begin{equation}\notag
    \begin{split}
        i \hbar \frac{\partial a(t)}{\partial t} & = [a(t), H] = \\
        & = \hbar \omega^{'}_a a(t) a^{\dagger}(t) a(t) + \hbar \int \omega^{'}_{R}(k) a(t) b^{\dagger}(k,t) b(k,t) d^{D}(k) + \\
        & + \hbar \lambda \int \frac{(a(t) b^{\dagger}(k,t) a(t) + a(t) a^{\dagger}(t) b(k,t))}{\sqrt{k^{2} + \alpha^{2}}} d^{D}(k) - \\
        & - \hbar \omega^{'}_a a^{\dagger}(t) a(t) a(t) - \hbar \int \omega^{'}_{R}(k) b^{\dagger}(k,t) b(k,t) a(t) d^{D}(k) - \\
        & - \hbar \lambda \int \frac{(b^{\dagger}(k,t) a(t) a(t) + a^{\dagger}(t) b(k,t) a(t))}{\sqrt{k^{2} + \alpha^{2}}} d^{D}(k) = \\
        & = xx' + yy' + zz' = \hbar \omega^{'}_a a(t) + \hbar \lambda \int \frac{b(k,t)}{\sqrt{k^{2} + \alpha^{2}}} d^{D}(k)
    \end{split}
\end{equation}

где
\begin{equation}\notag
    \begin{split}
        xx' & = \hbar \omega^{'}_a a(t) a^{\dagger}(t) a(t) - \hbar \omega^{'}_a a^{\dagger}(t) a(t) a(t) = \\
        & = \hbar \omega^{'}_a \underbrace{[a(t), a^{\dagger}(t)]}_{1} a(t)
          + \cancel{\hbar \omega^{'}_a a^{\dagger}(t) a(t) a(t)}
          - \cancel{\hbar \omega^{'}_a a^{\dagger}(t) a(t) a(t)} = \\
        & = \hbar \omega^{'}_a a(t) \\
        yy' & = \hbar \int \omega^{'}_{R}(k) \overbrace{a(t) b^{\dagger}(k,t) b(k,t)}^{= b^{\dagger}(k,t) a(t) b(k,t)} d^{D}(k) \\
        & - \hbar \int \omega^{'}_{R}(k) \overbrace{b^{\dagger}(k,t) b(k,t) a(t)}^{= b^{\dagger}(k,t) a(t) b(k,t)} d^{D}(k) = 0 \\
        zz' & = \hbar \lambda \int \frac{(a(t) b^{\dagger}(k,t) a(t)
        + a(t) a^{\dagger}(t) b(k,t))}{\sqrt{k^{2} + \alpha^{2}}} d^{D}(k) - \\
        & - \hbar \lambda \int \frac{(b^{\dagger}(k,t) a(t) a(t)
        + a^{\dagger}(t) b(k,t) a(t))}{\sqrt{k^{2} + \alpha^{2}}} d^{D}(k) = \\
    \end{split}
\end{equation}
\begin{equation}\notag
    \begin{split}
        zz' & = \hbar \lambda \int \frac{\overbrace{[a(t), b^{\dagger}(k,t)]}^{0} a(t) + \cancel{b^{\dagger}(k,t) a(t) a(t)}}{\sqrt{k^{2} + \alpha^{2}}} d^{D}(k) + \\
        & + \hbar \lambda \int \frac{[a(t), a^{\dagger}(t)] b(k,t) + \cancel{a^{\dagger}(t) a(t) b(k,t)}}{\sqrt{k^{2} + \alpha^{2}}} d^{D}(k) - \\
        & - \hbar \lambda \int \frac{\cancel{b^{\dagger}(k,t) a(t) a(t)} - a^{\dagger}(t) \overbrace{[a(t), b(k,t)]}^{0} +
            \cancel{a^{\dagger}(t) a(t) b(k,t)} }{\sqrt{k^{2} + \alpha^{2}}} d^{D}(k) = \\
        & = \hbar \lambda \int \frac{[a(t), a^{\dagger}(t)] b(k,t)}{\sqrt{k^{2} + \alpha^{2}}} d^{D}(k) =
        \hbar \lambda \int \frac{b(k,t)}{\sqrt{k^{2} + \alpha^{2}}} d^{D}(k)
    \end{split}
\end{equation}

Уравнения для динамкики мод можно перепиать в более простом виде:
{\large
\begin{equation}
    \begin{cases}
        & \dot{a}(t) = -i \omega^{'}_a a(t) - i \lambda \int \frac{b(k,t)}{\sqrt{k^{2} + \alpha^{2}}} d^{D}(k) \\
        & \dot{b}(k,t) = -i \omega^{'}_{R}(k) b(k,t) -i \lambda \frac{a(t)}{\sqrt{k^{2} + \alpha^{2}}}
    \end{cases}
\end{equation}}

Решение уравнения для динамики оператора $b(k,t)$ решается следующим образом. Сначала необходимо
решить однородное дифференциальное уравнение, которое для оператора $b(k,t)$ имеет вид:
\begin{equation}\notag
    \dot{b}(k,t) = -i \omega^{'}_{R}(k) b(k,t)
\end{equation}
Решение ЛОДУ довольно простое:
\begin{equation}\notag
    \begin{split}
        & \frac{\partial b(k,t)}{\partial t} = -i \omega^{'}_{R}(k) b(k,t) \\
        & \int \frac{\partial b(k,t)}{b(k,t)} = \int -i \omega^{'}_{R}(k) dt \\
        & ln b(k,t) = -i \omega^{'}_{R}(k) t + ln C \\
        & b(k,t) = Ce^{-i \omega^{'}_{R}(k) t} \\
    \end{split}
\end{equation}

Приняв константу $C = C(t)$, переходим к линейному неоднородному дифференциальному уравнению.
Чтобы найти $C(t)$ нужно подставить ЛНДУ в исходное уравнение:
\begin{equation}\notag
    \begin{split}
        & (C(t)e^{-i \omega^{'}_{R}(k) t})^{'}_{t} = -i \omega^{'}_{R}(k) C(t)e^{-i \omega^{'}_{R}(k) t} -i \lambda \frac{a(t)}{\sqrt{k^{2} + \alpha^{2}}} \\
        & C'(t)e^{-i \omega^{'}_{R}(k) t} - \cancel{i \omega^{'}_{R}(k) C(t)e^{-i \omega^{'}_{R}(k) t}} = \cancel{-i \omega^{'}_{R}(k) C(t)e^{-i \omega^{'}_{R}(k) t}} -i \lambda \frac{a(t)}{\sqrt{k^{2} + \alpha^{2}}} \\
        & C'(t)e^{-i \omega^{'}_{R}(k) t} = -i \lambda \frac{a(t)}{\sqrt{k^{2} + \alpha^{2}}} \\
        & C(t) = -i \lambda \int \frac{a(t)e^{i \omega^{'}_{R}(k) t}}{\sqrt{k^{2} + \alpha^{2}}} dt + C_{1}
    \end{split}
\end{equation}

Найдя $C(t)$, подставляем его в решение ЛНДУ:
\begin{equation}\notag
    \begin{split}
        & b(k,t) = -i \lambda \int_{0}^{t} \frac{a(\tau)e^{i \omega^{'}_{R}(k) \tau}e^{-i \omega^{'}_{R}(k) t}}{\sqrt{k^{2} + \alpha^{2}}} d\tau + C_{1}e^{-i \omega^{'}_{R}(k) t} \\
        & b(k,t) = -i \lambda \int_{0}^{t} \frac{a(\tau)e^{-i \omega^{'}_{R}(k) (t-\tau)}}{\sqrt{k^{2} + \alpha^{2}}} d\tau + C_{1}e^{-i \omega^{'}_{R}(k) t} \\
        & b(k,0) = b(k) = C_{1} \\
        & b(k,t) = b(k)e^{-i \omega^{'}_{R}(k) t} -i \lambda \int_{0}^{t} \frac{a(\tau)e^{-i \omega^{'}_{R}(k) (t-\tau)}}{\sqrt{k^{2} + \alpha^{2}}} d\tau \\
    \end{split}
\end{equation}

Проверка решения:
\begin{multline}\notag
    \Biggl( b(k)e^{-i \omega^{'}_{R}(k) t} -i \lambda \int_{0}^{t} \frac{a(\tau)e^{-i \omega^{'}_{R}(k) (t-\tau)}}{\sqrt{k^{2} + \alpha^{2}}} d\tau \Biggr)^{'}_{t} =  \\
    = b(k) \Bigl( e^{-i \omega^{'}_{R}(k) t} \Bigr)^{'}_{t} - i \lambda \Biggl( \int_{0}^{t} \frac{a(\tau)e^{-i \omega^{'}_{R}(k) (t-\tau)}}{\sqrt{k^{2} + \alpha^{2}}} d\tau \Biggr)^{'}_{t} = \\
    = -i \omega^{'}_{R}(k) \underbrace{b(k)e^{-i \omega^{'}_{R}(k) t}}_{b(k,0) = b(k,t)} -i \lambda \frac{a(t) \cancel{e^{-i \omega^{'}_{R}(k) (t-t)}}}{\sqrt{k^{2} + \alpha^{2}}} = \\
    = -i \omega^{'}_{R}(k) b(k,t) -i \lambda \frac{a(t)}{\sqrt{k^{2} + \alpha^{2}}}
\end{multline}

Приняв $\omega^{'}_{R}(k) = \omega k$, а также зная что:
\begin{equation}\notag
    \int \frac{e^{-i \omega k (t - \tau)}}{k^{2} + \alpha^{2}} dk = \frac{\pi}{\alpha} e^{-\alpha \omega (t - \tau)}
\end{equation}
подставляем полученное решение уравнения для $b(k,t)$ в уравнение $\dot{a}(t)$:
\begin{equation}
    \dot{a}(t) =
        -i \omega^{'}_{a} a(t)
        -i \lambda \int \frac{b(k)e^{-i \omega k t}}{\sqrt{k^{2} + \alpha^2}} dk
        -\frac{\lambda^{2} \pi}{\alpha} \int_{0}^{t} a(\tau) e^{- \alpha \omega (t - \tau)} d\tau
\end{equation}

Непосредственно получить решение этого уравнения не получится, поскольку последнее слагаемое
стоит под знаком интеграла. Но можно получить численное решение уравнения оператора принятия
решения $n_{a}(t)$, используя правило дифференцирования:
\begin{equation}\label{dot_na}
    \frac{\partial n_{a}(t)}{\partial t}
    = \frac{\partial (a^{\dagger}(t) a(t))}{\partial t}
    = \frac{\partial a^{\dagger}(t)}{\partial t}a(t) + a^{\dagger}(t)\frac{\partial a(t)}{\partial t}
    = \dot{a}^{\dagger}(t) a(t) + a^{\dagger}(t) \dot{a}(t)
\end{equation}
где
\begin{multline}\notag
    \dot{a}^{\dagger}(t) a(t) =
    i \omega^{'}_{a} a^{\dagger}(t) a(t) +
    i \lambda \int \frac{b^{\dagger}(k) a(t) e^{i \omega k t}}{\sqrt{k^{2} + \alpha^2}} dk \\
    - \frac{\lambda^{2} \pi}{\alpha} \int_{0}^{t} a^{\dagger}(\tau) a(t) e^{- \alpha \omega (t - \tau)} d\tau
\end{multline}
\begin{multline}\notag
    a^{\dagger}(t) \dot{a}(t) =
    -i \omega^{'}_{a} a^{\dagger}(t) a(t)
    -i \lambda \int \frac{a^{\dagger}(t) b(k)e^{-i \omega k t}}{\sqrt{k^{2} + \alpha^2}} dk \\
    - \frac{\lambda^{2} \pi}{\alpha} \int_{0}^{t} a^{\dagger}(t) a(\tau) e^{- \alpha \omega (t - \tau)} d\tau
\end{multline}
перепишем уравнение \eqref{dot_na} в простом виде:
\begin{multline}
    \frac{\partial n_{a}(t)}{\partial t} =
    i \lambda \int \frac{(b^{\dagger}(k) a(t) e^{i \omega k t} - a^{\dagger}(t) b(k)e^{-i \omega k t})}{\sqrt{k^{2} + \alpha^2}} dk - \\
    - \frac{\lambda^{2} \pi}{\alpha} \int_{0}^{t} e^{- \alpha \omega (t - \tau)} \overbrace{(a^{\dagger}(\tau) a(t) + a^{\dagger}(t) a(\tau))}^{n_{a}(t) \delta(t-\tau) + n_{a}(t) \delta(t-\tau)} d\tau
\end{multline}
в более простом виде \eqref{dot_na} выглядит следующим образом:
\begin{equation}
    \frac{\partial n_{a}(t)}{\partial t} =
    i \lambda \int \frac{(b^{\dagger}(k) a(t) e^{i \omega k t} - a^{\dagger}(t) b(k) e^{-i \omega k t})}{\sqrt{k^{2} + \alpha^2}} dk
    - 2 \frac{\lambda^{2} \pi}{\alpha} n_{a}(t)
\end{equation}
для нахождения произведений $b^{\dagger}(k) a(t)$ и $a^{\dagger}(t) b(k)$, можно их продифференцировать:
\begin{multline}\label{bcrossk_dota}
    \frac{\partial (b^{\dagger}(k) a(t))}{\partial t} =
    b^{\dagger}(k) \dot{a}(t) =
    -i \omega^{'}_{a} b^{\dagger}(k) a(t) - \\
    -i \lambda \int \frac{b^{\dagger}(k) b(k) e^{-i \omega k t}}{\sqrt{k^{2} + \alpha^2}} dk
    -\frac{\lambda^{2} \pi}{\alpha} \int_{0}^{t} b^{\dagger}(k) a(\tau) e^{- \alpha \omega (t - \tau)} d\tau
\end{multline}
\begin{multline}\label{dota_bk}
    \frac{\partial (a^{\dagger}(t) b(k))}{\partial t} =
    \dot{a}^{\dagger}(t) b(k) =
    i \omega^{'}_{a} a^{\dagger}(t) b(k) + \\
    + i \lambda \int \frac{b^{\dagger}(k) b(k) e^{i \omega k t}}{\sqrt{k^{2} + \alpha^2}} dk
    - \frac{\lambda^{2} \pi}{\alpha} \int_{0}^{t} a^{\dagger}(\tau) b(k) e^{- \alpha \omega (t - \tau)} d\tau
\end{multline}
принимая во внимание что $b^{\dagger}(k) b(k) = n_{b} \delta(k-k')$, то уравнения \eqref{bcrossk_dota} и \eqref{dota_bk}
имеют следующий вид:
\begin{equation}\label{bcrossk_dota_less}
    b^{\dagger}(k) \dot{a}(t) =
    -i \omega^{'}_{a} b^{\dagger}(k) a(t)
    -i \lambda \frac{n_{b} e^{-i \omega k t}}{\sqrt{k^{2} + \alpha^2}}
    -\frac{\lambda^{2} \pi}{\alpha} \int_{0}^{t} b^{\dagger}(k) a(\tau) e^{- \alpha \omega (t - \tau)} d\tau
\end{equation}
\begin{equation}\label{dota_bk_less}
    \dot{a}^{\dagger}(t) b(k) =
    i \omega^{'}_{a} a^{\dagger}(t) b(k)
    + i \lambda \frac{n_{b} e^{i \omega k t}}{\sqrt{k^{2} + \alpha^2}}
    - \frac{\lambda^{2} \pi}{\alpha} \int_{0}^{t} a^{\dagger}(\tau) b(k) e^{- \alpha \omega (t - \tau)} d\tau
\end{equation}
зная правило приращения \[f'(x) = \lim_{\Delta x \to 0} \frac{f(x + \Delta x) - f(x)}{\Delta x}\] можно получить
значения оператора принятия решения $n_{a}(t)$, используя итерационный метод и уравнения \eqref{bcrossk_dota_less},
\eqref{dota_bk_less}, где для упрощения $R_{1}(k,t) = b^{\dagger}(k) a(t)$, $R_{2}(k,t) = a^{\dagger}(t) b(k)$:
\begin{multline}
    n_{a}(t + \Delta t) =
    \Biggl[i \lambda \int \frac{R_{1}(k,t) e^{i \omega k t} - R_{2}(k,t) e^{-i \omega k t}}{\sqrt{k^{2} + \alpha^2}} dk - \\
    - 2 \frac{\lambda^{2} \pi}{\alpha} n_{a}(t) \Biggr] \Delta t + n_{a}(t)
\end{multline}
\begin{multline}
    R_{1}(k,t + \Delta t) =
    \Biggl[-i \omega^{'}_{a} R_{1}(k,t)
    -i \lambda \frac{n_{b} e^{-i \omega k t}}{\sqrt{k^{2} + \alpha^2}} - \\
    -\frac{\lambda^{2} \pi}{\alpha} \int_{0}^{t} R_{1}(k,\tau) e^{- \alpha \omega (t - \tau)} d\tau \Biggr] \Delta t + R_{1}(k,t)
\end{multline}
\begin{multline}
    R_{2}(k,t + \Delta t) =
    \Biggl[i \omega^{'}_{a} R_{2}(k,t)
    + i \lambda \frac{n_{b} e^{i \omega k t}}{\sqrt{k^{2} + \alpha^2}} + \\
    - \frac{\lambda^{2} \pi}{\alpha} \int_{0}^{t} R_{2}(k,\tau) e^{- \alpha \omega (t - \tau)} d\tau \Biggr] \Delta t + R_{2}(k,t)
\end{multline}
    %! Author = jaroslav
%! Date = 18.05.20
\chapter{Исходный код}

\lstinputlisting[breaklines=true, caption=n.py]{/home/jaroslav/Documents/lamp_calc_py/n.py}
\lstinputlisting[breaklines=true, caption=g.py]{/home/jaroslav/Documents/lamp_calc_py/g.py}
\lstinputlisting[breaklines=true, caption=R.py]{/home/jaroslav/Documents/lamp_calc_py/R.py}
\lstinputlisting[breaklines=true, caption=a.py]{/home/jaroslav/Documents/lamp_calc_py/a.py}
\lstinputlisting[breaklines=true, caption=test.py]{/home/jaroslav/Documents/lamp_calc_py/test_g.py}
\lstinputlisting[breaklines=true, caption=main.py]{/home/jaroslav/Documents/lamp_calc_py/main.py}
\lstinputlisting[breaklines=true, caption=n0.py]{/home/jaroslav/Documents/lamp_calc_py/n0_a.py}

\end{document}