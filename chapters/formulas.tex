%! Author = jaroslav
%! Date = 03.03.20
\chapter{Основные математические модели }

Задан гамильтониан эволюции:
\begin{equation}\label{hf}
    H_f = \hbar \omega_a a^{\dagger} a,
\end{equation}
Задан гамильтониан, описывающий резервуар:
\begin{equation}\label{hr}
    H_R = \hbar \int \omega_{R}(k) b^{\dagger}(k) b(k) d^{D}(k),
\end{equation}
Задан гамильтониан взимодействия резервуара и моды:
\begin{equation}\label{hint}
    H_{int} = \hbar \lambda \int g(k) (b^{\dagger}(k) a + a^{\dagger} b(k)) d^{D}(k),
\end{equation}
Для данной системы выполняются следующие коммутационные соотношения:
$
[\hat{a}, \hat{a}^{\dagger}] = 1,
[\hat{b}(k), \hat{b}^{\dagger}(k)] = \delta (k-k')
$
Приняв для данной системы ядро вида:
\begin{equation}
    g^{2}(k) = \frac{g}{k^{2} + \alpha^{2}},
\end{equation}
и отнормировав по $g$ для уравнений (\ref{hf}-\ref{hint}), получаем следующие выражения:
\begin{gather}
    H_f = \hbar \omega^{'}_a a^{\dagger} a, \\
    H_R = \hbar \int \omega^{'}_{R}(k) b^{\dagger}(k) b(k) d^{D}(k), \\
    H_{int} = \hbar \lambda \int \frac{(b^{\dagger}(k) a + a^{\dagger} b(k))}{\sqrt{k^{2} + \alpha^{2}}} d^{D}(k),
\end{gather}
где $\omega^{'}_a = \omega_a/g$, $\omega^{'}_{R}(k) = \omega_{R}(k)/g$

Полный гамильтониан системы состоит из суммы этих уравнений:
\begin{multline}
    H = H_f + H_R + H_{int} = \hbar \omega^{'}_a a^{\dagger} a + \hbar \int \omega^{'}_{R}(k) b^{\dagger}(k) b(k) d^{D}(k) \\
    + \hbar \lambda \int \frac{(b^{\dagger}(k) a + a^{\dagger} b(k))}{\sqrt{k^{2} + \alpha^{2}}} d^{D}(k)
\end{multline}
Зная коммутационные соотношения и полный гамильтониан системы можно решить уравнение Гейзенберга
для операторов $a$ и $b(k)$. Для оператора $b(k)$ будет следующее решение:
\begin{equation}\notag
    \begin{split}
        i \hbar \frac{\partial b(k,t)}{\partial t} & = [b(k,t), H] = \\
        & = \hbar \omega^{'}_a b(k,t) a^{\dagger}(t) a(t) + \hbar \int \omega^{'}_{R}(k) b(k,t) b^{\dagger}(k,t) b(k,t) d^{D}(k) + \\
        & + \hbar \lambda \int \frac{(b(k,t) b^{\dagger}(k,t) a(t) + b(k,t) a^{\dagger}(t) b(k,t))}{\sqrt{k^{2} + \alpha^{2}}} d^{D}(k) - \\
        & - \hbar \omega^{'}_a a^{\dagger}(t) a(t) b(k,t) - \hbar \int \omega^{'}_{R}(k) b^{\dagger}(k,t) b(k,t) b(k,t) d^{D}(k) - \\
        & - \hbar \lambda \int \frac{(b^{\dagger}(k,t) a(t) b(k,t) + a^{\dagger}(t) b(k,t) b(k,t))}{\sqrt{k^{2} + \alpha^{2}}} d^{D}(k) = \\
        & = xx' + yy' + zz' = \hbar \omega^{'}_{R}(k) b(k,t) + \hbar \lambda \frac{a(t)}{\sqrt{k^{2} + \alpha^{2}}}
    \end{split}
\end{equation}
где
\begin{equation}\notag
    \begin{split}
        xx' & = \hbar \omega^{'}_a b(k,t) a^{\dagger}(t) a(t) - \hbar \omega^{'}_a a^{\dagger}(t) a(t) b(k,t) = \\
        & = \cancel{\hbar \omega^{'}_a a^{\dagger}(t) b(k,t) a(t)} - \hbar \omega^{'}_a \underbrace{[a^{\dagger}(t), b(k,t)]}_{0} a(t) - \\
        & - \hbar \omega^{'}_a a^{\dagger}(t) \underbrace{[a(t), b(k,t)]}_{0} - \cancel{\hbar \omega^{'}_a a^{\dagger}(t) b(k,t) a(t)} = \\
        & = 0
    \end{split}
\end{equation}
\begin{equation}\notag
    \begin{split}
        yy' & = \hbar \int \omega^{'}_{R}(k) b(k,t) b^{\dagger}(k,t) b(k,t) d^{D}(k) - \\
        & - \hbar \int \omega^{'}_{R}(k) b^{\dagger}(k,t) b(k,t) b(k,t) d^{D}(k) = \\
        & = \hbar \int \omega^{'}_{R}(k) \Bigl([b(k,t), b^{\dagger}(k,t)] b(k,t) + \\
        & + \cancel{b^{\dagger}(k,t) b(k,t) b(k,t)} - \cancel{b^{\dagger}(k,t) b(k,t) b(k,t)} \Bigr) d^{D}(k) = \\
        & = \hbar \int \omega^{'}_{R}(k) \delta (k-k') b(k,t) d^{D}(k) = \hbar \omega^{'}_{R}(k') b(k',t)
    \end{split}
\end{equation}
\begin{equation}\notag
    \begin{split}
        zz' & = \hbar \lambda \int \frac{(b(k,t) b^{\dagger}(k,t) a(t) + b(k,t) a^{\dagger}(t) b(k,t))}{\sqrt{k^{2} + \alpha^{2}}} d^{D}(k) - \\
        & - \hbar \lambda \int \frac{(b^{\dagger}(k,t) a(t) b(k,t) + a^{\dagger}(t) b(k,t) b(k,t))}{\sqrt{k^{2} + \alpha^{2}}} d^{D}(k) = \\
        & = \hbar \lambda \int \frac{[b(k,t), b^{\dagger}(k,t)] a(t) + \cancel{b^{\dagger}(k,t) b(k,t) a(t)}}{\sqrt{k^{2} + \alpha^{2}}} d^{D}(k) - \\
        & - \hbar \lambda \int \frac{\overbrace{[a^{\dagger}(t), b(k,t)]}^{0} b(k,t) + \cancel{a^{\dagger}(t) b(k,t) b(k,t)}}{\sqrt{k^{2} + \alpha^{2}}} d^{D}(k) - \\
        & - \hbar \lambda \int \frac{b^{\dagger}(k,t) \overbrace{[a(t), b(k,t)]}^{0} + \cancel{b^{\dagger}(k,t) b(k,t) a(t)} + }{\sqrt{k^{2} + \alpha^{2}}} d^{D}(k) = \\
        & + \hbar \lambda \int \frac{\cancel{a^{\dagger}(t) b(k,t) b(k,t)}}{\sqrt{k^{2} + \alpha^{2}}} d^{D}(k) = \\
        & = \hbar \lambda \int \frac{[b(k,t), b^{\dagger}(k,t)] a(t)}{\sqrt{k^{2} + \alpha^{2}}} d^{D}(k) = \hbar \lambda \int \frac{\delta (k-k') a(t)}{\sqrt{k^{2} + \alpha^{2}}} d^{D}(k) = \\
        & = \hbar \lambda \frac{a(t)}{\sqrt{k^{2} + \alpha^{2}}}
    \end{split}
\end{equation}

Для оператора $a$ решение:
\begin{equation}\notag
    \begin{split}
        i \hbar \frac{\partial a(t)}{\partial t} & = [a(t), H] = \\
        & = \hbar \omega^{'}_a a(t) a^{\dagger}(t) a(t) + \hbar \int \omega^{'}_{R}(k) a(t) b^{\dagger}(k,t) b(k,t) d^{D}(k) + \\
        & + \hbar \lambda \int \frac{(a(t) b^{\dagger}(k,t) a(t) + a(t) a^{\dagger}(t) b(k,t))}{\sqrt{k^{2} + \alpha^{2}}} d^{D}(k) - \\
        & - \hbar \omega^{'}_a a^{\dagger}(t) a(t) a(t) - \hbar \int \omega^{'}_{R}(k) b^{\dagger}(k,t) b(k,t) a(t) d^{D}(k) - \\
        & - \hbar \lambda \int \frac{(b^{\dagger}(k,t) a(t) a(t) + a^{\dagger}(t) b(k,t) a(t))}{\sqrt{k^{2} + \alpha^{2}}} d^{D}(k) = \\
        & = xx' + yy' + zz' = \hbar \omega^{'}_a a(t) + \hbar \lambda \int \frac{b(k,t)}{\sqrt{k^{2} + \alpha^{2}}} d^{D}(k)
    \end{split}
\end{equation}

где
\begin{equation}\notag
    \begin{split}
        xx' & = \hbar \omega^{'}_a a(t) a^{\dagger}(t) a(t) - \hbar \omega^{'}_a a^{\dagger}(t) a(t) a(t) = \\
        & = \hbar \omega^{'}_a \underbrace{[a(t), a^{\dagger}(t)]}_{1} a(t)
          + \cancel{\hbar \omega^{'}_a a^{\dagger}(t) a(t) a(t)}
          - \cancel{\hbar \omega^{'}_a a^{\dagger}(t) a(t) a(t)} = \\
        & = \hbar \omega^{'}_a a(t) \\
        yy' & = \hbar \int \omega^{'}_{R}(k) \overbrace{a(t) b^{\dagger}(k,t) b(k,t)}^{= b^{\dagger}(k,t) a(t) b(k,t)} d^{D}(k) \\
        & - \hbar \int \omega^{'}_{R}(k) \overbrace{b^{\dagger}(k,t) b(k,t) a(t)}^{= b^{\dagger}(k,t) a(t) b(k,t)} d^{D}(k) = 0 \\
        zz' & = \hbar \lambda \int \frac{(a(t) b^{\dagger}(k,t) a(t)
        + a(t) a^{\dagger}(t) b(k,t))}{\sqrt{k^{2} + \alpha^{2}}} d^{D}(k) - \\
        & - \hbar \lambda \int \frac{(b^{\dagger}(k,t) a(t) a(t)
        + a^{\dagger}(t) b(k,t) a(t))}{\sqrt{k^{2} + \alpha^{2}}} d^{D}(k) = \\
    \end{split}
\end{equation}
\begin{equation}\notag
    \begin{split}
        zz' & = \hbar \lambda \int \frac{\overbrace{[a(t), b^{\dagger}(k,t)]}^{0} a(t) + \cancel{b^{\dagger}(k,t) a(t) a(t)}}{\sqrt{k^{2} + \alpha^{2}}} d^{D}(k) + \\
        & + \hbar \lambda \int \frac{[a(t), a^{\dagger}(t)] b(k,t) + \cancel{a^{\dagger}(t) a(t) b(k,t)}}{\sqrt{k^{2} + \alpha^{2}}} d^{D}(k) - \\
        & - \hbar \lambda \int \frac{\cancel{b^{\dagger}(k,t) a(t) a(t)} - a^{\dagger}(t) \overbrace{[a(t), b(k,t)]}^{0} +
            \cancel{a^{\dagger}(t) a(t) b(k,t)} }{\sqrt{k^{2} + \alpha^{2}}} d^{D}(k) = \\
        & = \hbar \lambda \int \frac{[a(t), a^{\dagger}(t)] b(k,t)}{\sqrt{k^{2} + \alpha^{2}}} d^{D}(k) =
        \hbar \lambda \int \frac{b(k,t)}{\sqrt{k^{2} + \alpha^{2}}} d^{D}(k)
    \end{split}
\end{equation}

Уравнения для динамкики мод можно перепиать в более простом виде:
{\large
\begin{equation}
    \begin{cases}
        & \dot{a}(t) = -i \omega^{'}_a a(t) - i \lambda \int \frac{b(k,t)}{\sqrt{k^{2} + \alpha^{2}}} d^{D}(k) \\
        & \dot{b}(k,t) = -i \omega^{'}_{R}(k) b(k,t) -i \lambda \frac{a(t)}{\sqrt{k^{2} + \alpha^{2}}}
    \end{cases}
\end{equation}}

Решение уравнения для динамики оператора $b(k,t)$ решается следующим образом. Сначала необходимо
решить однородное дифференциальное уравнение, которое для оператора $b(k,t)$ имеет вид:
\begin{equation}\notag
    \dot{b}(k,t) = -i \omega^{'}_{R}(k) b(k,t)
\end{equation}
Решение ЛОДУ довольно простое:
\begin{equation}\notag
    \begin{split}
        & \frac{\partial b(k,t)}{\partial t} = -i \omega^{'}_{R}(k) b(k,t) \\
        & \int \frac{\partial b(k,t)}{b(k,t)} = \int -i \omega^{'}_{R}(k) dt \\
        & ln b(k,t) = -i \omega^{'}_{R}(k) t + ln C \\
        & b(k,t) = Ce^{-i \omega^{'}_{R}(k) t} \\
    \end{split}
\end{equation}

Приняв константу $C = C(t)$, переходим к линейному неоднородному дифференциальному уравнению.
Чтобы найти $C(t)$ нужно подставить ЛНДУ в исходное уравнение:
\begin{equation}\notag
    \begin{split}
        & (C(t)e^{-i \omega^{'}_{R}(k) t})^{'}_{t} = -i \omega^{'}_{R}(k) C(t)e^{-i \omega^{'}_{R}(k) t} -i \lambda \frac{a(t)}{\sqrt{k^{2} + \alpha^{2}}} \\
        & C'(t)e^{-i \omega^{'}_{R}(k) t} - \cancel{i \omega^{'}_{R}(k) C(t)e^{-i \omega^{'}_{R}(k) t}} = \cancel{-i \omega^{'}_{R}(k) C(t)e^{-i \omega^{'}_{R}(k) t}} -i \lambda \frac{a(t)}{\sqrt{k^{2} + \alpha^{2}}} \\
        & C'(t)e^{-i \omega^{'}_{R}(k) t} = -i \lambda \frac{a(t)}{\sqrt{k^{2} + \alpha^{2}}} \\
        & C(t) = -i \lambda \int \frac{a(t)e^{i \omega^{'}_{R}(k) t}}{\sqrt{k^{2} + \alpha^{2}}} dt + C_{1}
    \end{split}
\end{equation}

Найдя $C(t)$, подставляем его в решение ЛНДУ:
\begin{equation}\notag
    \begin{split}
        & b(k,t) = -i \lambda \int_{0}^{t} \frac{a(\tau)e^{i \omega^{'}_{R}(k) \tau}e^{-i \omega^{'}_{R}(k) t}}{\sqrt{k^{2} + \alpha^{2}}} d\tau + C_{1}e^{-i \omega^{'}_{R}(k) t} \\
        & b(k,t) = -i \lambda \int_{0}^{t} \frac{a(\tau)e^{-i \omega^{'}_{R}(k) (t-\tau)}}{\sqrt{k^{2} + \alpha^{2}}} d\tau + C_{1}e^{-i \omega^{'}_{R}(k) t} \\
        & b(k,0) = b(k) = C_{1} \\
        & b(k,t) = b(k)e^{-i \omega^{'}_{R}(k) t} -i \lambda \int_{0}^{t} \frac{a(\tau)e^{-i \omega^{'}_{R}(k) (t-\tau)}}{\sqrt{k^{2} + \alpha^{2}}} d\tau \\
    \end{split}
\end{equation}

Проверка решения:
\begin{multline}\notag
    \Biggl( b(k)e^{-i \omega^{'}_{R}(k) t} -i \lambda \int_{0}^{t} \frac{a(\tau)e^{-i \omega^{'}_{R}(k) (t-\tau)}}{\sqrt{k^{2} + \alpha^{2}}} d\tau \Biggr)^{'}_{t} =  \\
    = b(k) \Bigl( e^{-i \omega^{'}_{R}(k) t} \Bigr)^{'}_{t} - i \lambda \Biggl( \int_{0}^{t} \frac{a(\tau)e^{-i \omega^{'}_{R}(k) (t-\tau)}}{\sqrt{k^{2} + \alpha^{2}}} d\tau \Biggr)^{'}_{t} = \\
    = -i \omega^{'}_{R}(k) \underbrace{b(k)e^{-i \omega^{'}_{R}(k) t}}_{b(k,0) = b(k,t)} -i \lambda \frac{a(t) \cancel{e^{-i \omega^{'}_{R}(k) (t-t)}}}{\sqrt{k^{2} + \alpha^{2}}} = \\
    = -i \omega^{'}_{R}(k) b(k,t) -i \lambda \frac{a(t)}{\sqrt{k^{2} + \alpha^{2}}}
\end{multline}

Приняв $\omega^{'}_{R}(k) = \omega k$, а также зная что:
\begin{equation}\notag
    \int \frac{e^{-i \omega k (t - \tau)}}{k^{2} + \alpha^{2}} dk = \frac{\pi}{\alpha} e^{-\alpha \omega (t - \tau)}
\end{equation}
подставляем полученное решение уравнения для $b(k,t)$ в уравнение $\dot{a}(t)$:
\begin{equation}
    \dot{a}(t) =
        -i \omega^{'}_{a} a(t)
        -i \lambda \int \frac{b(k)e^{-i \omega k t}}{\sqrt{k^{2} + \alpha^2}} dk
        -\frac{\lambda^{2} \pi}{\alpha} \int_{0}^{t} a(\tau) e^{- \alpha \omega (t - \tau)} d\tau
\end{equation}

Непосредственно получить решение этого уравнения не получится, поскольку последнее слагаемое
стоит под знаком интеграла. Но можно получить численное решение уравнения оператора принятия
решения $n_{a}(t)$, используя правило дифференцирования:
\begin{equation}\label{dot_na}
    \frac{\partial n_{a}(t)}{\partial t}
    = \frac{\partial (a^{\dagger}(t) a(t))}{\partial t}
    = \frac{\partial a^{\dagger}(t)}{\partial t}a(t) + a^{\dagger}(t)\frac{\partial a(t)}{\partial t}
    = \dot{a}^{\dagger}(t) a(t) + a^{\dagger}(t) \dot{a}(t)
\end{equation}
где
\begin{multline}\notag
    \dot{a}^{\dagger}(t) a(t) =
    i \omega^{'}_{a} a^{\dagger}(t) a(t) +
    i \lambda \int \frac{b^{\dagger}(k) a(t) e^{i \omega k t}}{\sqrt{k^{2} + \alpha^2}} dk \\
    - \frac{\lambda^{2} \pi}{\alpha} \int_{0}^{t} a^{\dagger}(\tau) a(t) e^{- \alpha \omega (t - \tau)} d\tau
\end{multline}
\begin{multline}\notag
    a^{\dagger}(t) \dot{a}(t) =
    -i \omega^{'}_{a} a^{\dagger}(t) a(t)
    -i \lambda \int \frac{a^{\dagger}(t) b(k)e^{-i \omega k t}}{\sqrt{k^{2} + \alpha^2}} dk \\
    - \frac{\lambda^{2} \pi}{\alpha} \int_{0}^{t} a^{\dagger}(t) a(\tau) e^{- \alpha \omega (t - \tau)} d\tau
\end{multline}
перепишем уравнение \eqref{dot_na} в простом виде:
\begin{multline}
    \frac{\partial n_{a}(t)}{\partial t} =
    i \lambda \int \frac{(b^{\dagger}(k) a(t) e^{i \omega k t} - a^{\dagger}(t) b(k)e^{-i \omega k t})}{\sqrt{k^{2} + \alpha^2}} dk - \\
    - \frac{\lambda^{2} \pi}{\alpha} \int_{0}^{t} e^{- \alpha \omega (t - \tau)} \overbrace{(a^{\dagger}(\tau) a(t) + a^{\dagger}(t) a(\tau))}^{n_{a}(t) \delta(t-\tau) + n_{a}(t) \delta(t-\tau)} d\tau
\end{multline}
в более простом виде \eqref{dot_na} выглядит следующим образом:
\begin{equation}
    \frac{\partial n_{a}(t)}{\partial t} =
    i \lambda \int \frac{(b^{\dagger}(k) a(t) e^{i \omega k t} - a^{\dagger}(t) b(k) e^{-i \omega k t})}{\sqrt{k^{2} + \alpha^2}} dk
    - 2 \frac{\lambda^{2} \pi}{\alpha} n_{a}(t)
\end{equation}
для нахождения произведений $b^{\dagger}(k) a(t)$ и $a^{\dagger}(t) b(k)$, можно их продифференцировать:
\begin{multline}\label{bcrossk_dota}
    \frac{\partial (b^{\dagger}(k) a(t))}{\partial t} =
    b^{\dagger}(k) \dot{a}(t) =
    -i \omega^{'}_{a} b^{\dagger}(k) a(t) - \\
    -i \lambda \int \frac{b^{\dagger}(k) b(k) e^{-i \omega k t}}{\sqrt{k^{2} + \alpha^2}} dk
    -\frac{\lambda^{2} \pi}{\alpha} \int_{0}^{t} b^{\dagger}(k) a(\tau) e^{- \alpha \omega (t - \tau)} d\tau
\end{multline}
\begin{multline}\label{dota_bk}
    \frac{\partial (a^{\dagger}(t) b(k))}{\partial t} =
    \dot{a}^{\dagger}(t) b(k) =
    i \omega^{'}_{a} a^{\dagger}(t) b(k) + \\
    + i \lambda \int \frac{b^{\dagger}(k) b(k) e^{i \omega k t}}{\sqrt{k^{2} + \alpha^2}} dk
    - \frac{\lambda^{2} \pi}{\alpha} \int_{0}^{t} a^{\dagger}(\tau) b(k) e^{- \alpha \omega (t - \tau)} d\tau
\end{multline}
принимая во внимание что $b^{\dagger}(k) b(k) = n_{b} \delta(k-k')$, то уравнения \eqref{bcrossk_dota} и \eqref{dota_bk}
имеют следующий вид:
\begin{equation}\label{bcrossk_dota_less}
    b^{\dagger}(k) \dot{a}(t) =
    -i \omega^{'}_{a} b^{\dagger}(k) a(t)
    -i \lambda \frac{n_{b} e^{-i \omega k t}}{\sqrt{k^{2} + \alpha^2}}
    -\frac{\lambda^{2} \pi}{\alpha} \int_{0}^{t} b^{\dagger}(k) a(\tau) e^{- \alpha \omega (t - \tau)} d\tau
\end{equation}
\begin{equation}\label{dota_bk_less}
    \dot{a}^{\dagger}(t) b(k) =
    i \omega^{'}_{a} a^{\dagger}(t) b(k)
    + i \lambda \frac{n_{b} e^{i \omega k t}}{\sqrt{k^{2} + \alpha^2}}
    - \frac{\lambda^{2} \pi}{\alpha} \int_{0}^{t} a^{\dagger}(\tau) b(k) e^{- \alpha \omega (t - \tau)} d\tau
\end{equation}
зная правило приращения \[f'(x) = \lim_{\Delta x \to 0} \frac{f(x + \Delta x) - f(x)}{\Delta x}\] можно получить
значения оператора принятия решения $n_{a}(t)$, используя итерационный метод и уравнения \eqref{bcrossk_dota_less},
\eqref{dota_bk_less}, где для упрощения $R_{1}(k,t) = b^{\dagger}(k) a(t)$, $R_{2}(k,t) = a^{\dagger}(t) b(k)$:
\begin{multline}
    n_{a}(t + \Delta t) =
    \Biggl[i \lambda \int \frac{R_{1}(k,t) e^{i \omega k t} - R_{2}(k,t) e^{-i \omega k t}}{\sqrt{k^{2} + \alpha^2}} dk - \\
    - 2 \frac{\lambda^{2} \pi}{\alpha} n_{a}(t) \Biggr] \Delta t + n_{a}(t)
\end{multline}
\begin{multline}
    R_{1}(k,t + \Delta t) =
    \Biggl[-i \omega^{'}_{a} R_{1}(k,t)
    -i \lambda \frac{n_{b} e^{-i \omega k t}}{\sqrt{k^{2} + \alpha^2}} - \\
    -\frac{\lambda^{2} \pi}{\alpha} \int_{0}^{t} R_{1}(k,\tau) e^{- \alpha \omega (t - \tau)} d\tau \Biggr] \Delta t + R_{1}(k,t)
\end{multline}
\begin{multline}
    R_{2}(k,t + \Delta t) =
    \Biggl[i \omega^{'}_{a} R_{2}(k,t)
    + i \lambda \frac{n_{b} e^{i \omega k t}}{\sqrt{k^{2} + \alpha^2}} + \\
    - \frac{\lambda^{2} \pi}{\alpha} \int_{0}^{t} R_{2}(k,\tau) e^{- \alpha \omega (t - \tau)} d\tau \Biggr] \Delta t + R_{2}(k,t)
\end{multline}