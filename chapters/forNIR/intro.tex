%! Author = jaroslav
%! Date = 22.04.20
\intro

Современная наука хорошо справляется с естественнонаучными задачами окружающего мира.
Разработаны теории, модели которые хорошо описывают основные законы физики, химии, биологии.
И только к началу двадцатого века, ученые пришли к решению когнитивых задач.
Конечно, ранее эти задачи тоже изучались, но не так интенсивно и обсуждались лишь в
кругах филосовских мыслителей.
Такое позднее изучение когнитивных задач связано, конечно же, со сложностью формулировки,
постановкой экспериментов и малой применимостью результатов.
В современное время, необходимость в изучении когнитивных наук стало весомее из-за различных
эффектов с ошибками выбора правильного решения, развития задач искусственного интелекта,
изменений социальных отношений в современном обществе.
Как и любая другая наука, когнитивные задачи имеют свои парадоксы, которые сложно объяснить
простыми словами или законами.
В таком случае необходимо подбирать решения используя уже известные науке интрументы.
В данной работе такими инструментами являются теория графов и математический формализм.