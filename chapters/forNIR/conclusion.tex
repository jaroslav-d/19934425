%! Author = jaroslav
%! Date = 22.04.20
\conclusion

В~ходе~работы были изучены известные когнитивные эффекты происходящие в условиях неопределенности.
Присутствие неопределенности вносит искажения в мышление человека и приводит к нелогическим выводам.
Нелогические выводы скорее всего связаны с невозможностью удержать в мыслях все возможные исходы,
что как раз предполагают в своих выводах Канеман и Тверски~\citep{tversky1983extensional}.
Невозможность удержать все факты в уме подтверждается экспериментами Эббингауза, когда человек
постепенно забывал только что поступившую к нему информацию, а в некоторых случаях даже искажал~\citep{bartlett2002human}.
В экспериментах Эббингауза кривая забывания описывается такой же кривой как в модели
социально-значимых Интернет ресурсов, что подтверждает вляние памяти на отношение к некоторому утверждению.
Раз отношение человека к некоторому утверждению изменяется со временем из-за влияния эффектов памяти,
значит и при принятии решения у человека может меняться выбор.
В случае дилеммы заключенного варианты выбора являются противоречивыми, однако человек рассматривает
их единым целым, поскольку вопрос был задан один для любого варианта, из-за чего факты в пользу одного
варианта или другого могут накладываться друг на друга из-за эффекта памяти и таким образом запутываться.
Явление запутанности давно изучено в квантовой физике, благодаря квантовым эффектам наблюдаемым
в окружающем мире, но помимо этого стоит отметить хорошо развитый математический формализм,
благодаря которому можно описать эффекты запутывания.