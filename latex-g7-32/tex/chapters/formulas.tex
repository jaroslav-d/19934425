%! Author = jaroslav
%! Date = 03.03.20
\chapter{Основные аналитические методы построения модели}
\label{cha:appendix1}

Рассмотрим резервуар~с бесконечным числом степеней свободы $k$ частицы которого описываются~бесконечным~числом
оператора рождения(уничтожения) $b^\dagger(b)$.
Данный резервуар определяет информационную среду,~с~которой~взаимодействует агент ПР.
ПР агента в представлении вторичного~квантования характеризуется операторами рождения (уничтожения)
$a^\dagger(a)$ в Гильбертовом пространстве.
Эти операторы должны подчиняться следующим коммутационным соотношениям:
$
[\hat{a}, \hat{a}^{\dagger}] = 1,
[\hat{b}(k), \hat{b}^{\dagger}(k)] = \delta (k-k')
$
в то время как все остальные коммутаторы предполагаются равными нулю.
Полный Гамильтониан расматриваемой системы взаимодействия внешнего информационного поля (резервуара) и агента ПР имеет вид:

\begin{multline}\label{H}
H_f = \hbar \omega_a a^{\dagger} a + \hbar \lambda \int g(k) (b^{\dagger}(k) a + a^{\dagger} b(k)) d^{D}(k) \\
+ \hbar \lambda \int g(k) (b^{\dagger}(k) a + a^{\dagger} b(k)) d^{D}(k),
\end{multline}

где ${\omega_a}$($\omega_b$) - частотная характеристика~агента(резервуара) соотвественно; $\lambda$
- коэффициент взаимодействия агента и резервуара; $g(k)$ - ядро, определеяющее сруктуру резервуара;
$D$ - размерность пространства резервуара.
В данной работе мы ограничеваемся одномерным резервуаром $D=1$.
Ядро резервуара $g(k)$ определяющее взаимодействие информационной среды и агента ПР в данной работе имеет вид:

\begin{equation}
    g^{2}(k) = \frac{g}{k^{2} + \alpha^{2}},
\end{equation}

Таким~образом, отнормировав~уравнение \eqref{H} на константу $g$, определяющая ширину распределения
ядра мы получим:

\begin{equation}\label{H'}
H' =  \omega^{'}_a a^{\dagger} a +  \int \omega^{'}_{R}(k) b^{\dagger}(k) b(k) dk+ \lambda \int \frac{(b^{\dagger}(k) a + a^{\dagger} b(k))}{\sqrt{k^{2} + \alpha^{2}}} dk,
\end{equation}
где $\omega^{'}_a = \omega_a/g$, $\omega^{'}_{R}(k) = \omega_{R}(k)/g$.

Решая уравнение Гейзенберга  $i\frac{\partial f}{\partial t}=[f,H]$, $ f \in a(a^{\dagger})$ и
$b(b^{\dagger})$ можно получить операторные уравнения для операторов уничтожения $a$ и $b$.
Принимая во внимание коммутационные соотношение для операторов $a$ и $b$, решение уравнение
Гейзенберга для оператора $b$ имеет вид:

\begin{equation}\notag
\begin{split}
    i \hbar \frac{\partial b(k,t)}{\partial t} & = [b(k,t), H] = \\
    & = \hbar \omega^{'}_a b(k,t) a^{\dagger}(t) a(t) + \hbar \int \omega^{'}_{R}(k) b(k,t) b^{\dagger}(k,t) b(k,t) d^{D}(k) + \\
    & + \hbar \lambda \int \frac{(b(k,t) b^{\dagger}(k,t) a(t) + b(k,t) a^{\dagger}(t) b(k,t))}{\sqrt{k^{2} + \alpha^{2}}} d^{D}(k) - \\
    & - \hbar \omega^{'}_a a^{\dagger}(t) a(t) b(k,t) - \hbar \int \omega^{'}_{R}(k) b^{\dagger}(k,t) b(k,t) b(k,t) d^{D}(k) - \\
    & - \hbar \lambda \int \frac{(b^{\dagger}(k,t) a(t) b(k,t) + a^{\dagger}(t) b(k,t) b(k,t))}{\sqrt{k^{2} + \alpha^{2}}} d^{D}(k) = \\
    & = xx' + yy' + zz' = \hbar \omega^{'}_{R}(k) b(k,t) + \hbar \lambda \frac{a(t)}{\sqrt{k^{2} + \alpha^{2}}}
\end{split}
\end{equation}
где
\begin{equation}\notag
\begin{split}
    xx' & = \hbar \omega^{'}_a b(k,t) a^{\dagger}(t) a(t) - \hbar \omega^{'}_a a^{\dagger}(t) a(t) b(k,t) = \\
    & = \cancel{\hbar \omega^{'}_a a^{\dagger}(t) b(k,t) a(t)} - \hbar \omega^{'}_a \underbrace{[a^{\dagger}(t), b(k,t)]}_{0} a(t) - \\
    & - \hbar \omega^{'}_a a^{\dagger}(t) \underbrace{[a(t), b(k,t)]}_{0} - \cancel{\hbar \omega^{'}_a a^{\dagger}(t) b(k,t) a(t)} = \\
    & = 0
\end{split}
\end{equation}
\begin{equation}\notag
\begin{split}
    yy' & = \hbar \int \omega^{'}_{R}(k) b(k,t) b^{\dagger}(k,t) b(k,t) d^{D}(k) - \\
    & - \hbar \int \omega^{'}_{R}(k) b^{\dagger}(k,t) b(k,t) b(k,t) d^{D}(k) = \\
    & = \hbar \int \omega^{'}_{R}(k) \Bigl([b(k,t), b^{\dagger}(k,t)] b(k,t) + \\
    & + \cancel{b^{\dagger}(k,t) b(k,t) b(k,t)} - \cancel{b^{\dagger}(k,t) b(k,t) b(k,t)} \Bigr) d^{D}(k) = \\
    & = \hbar \int \omega^{'}_{R}(k) \delta (k-k') b(k,t) d^{D}(k) = \hbar \omega^{'}_{R}(k') b(k',t)
\end{split}
\end{equation}
\begin{equation}\notag
\begin{split}
    zz' & = \hbar \lambda \int \frac{(b(k,t) b^{\dagger}(k,t) a(t) + b(k,t) a^{\dagger}(t) b(k,t))}{\sqrt{k^{2} + \alpha^{2}}} d^{D}(k) - \\
    & - \hbar \lambda \int \frac{(b^{\dagger}(k,t) a(t) b(k,t) + a^{\dagger}(t) b(k,t) b(k,t))}{\sqrt{k^{2} + \alpha^{2}}} d^{D}(k) = \\
    & = \hbar \lambda \int \frac{[b(k,t), b^{\dagger}(k,t)] a(t) + \cancel{b^{\dagger}(k,t) b(k,t) a(t)}}{\sqrt{k^{2} + \alpha^{2}}} d^{D}(k) - \\
    & - \hbar \lambda \int \frac{\overbrace{[a^{\dagger}(t), b(k,t)]}^{0} b(k,t) + \cancel{a^{\dagger}(t) b(k,t) b(k,t)}}{\sqrt{k^{2} + \alpha^{2}}} d^{D}(k) - \\
    & - \hbar \lambda \int \frac{b^{\dagger}(k,t) \overbrace{[a(t), b(k,t)]}^{0} + \cancel{b^{\dagger}(k,t) b(k,t) a(t)} + }{\sqrt{k^{2} + \alpha^{2}}} d^{D}(k) = \\
    & + \hbar \lambda \int \frac{\cancel{a^{\dagger}(t) b(k,t) b(k,t)}}{\sqrt{k^{2} + \alpha^{2}}} d^{D}(k) = \\
    & = \hbar \lambda \int \frac{[b(k,t), b^{\dagger}(k,t)] a(t)}{\sqrt{k^{2} + \alpha^{2}}} d^{D}(k) = \hbar \lambda \int \frac{\delta (k-k') a(t)}{\sqrt{k^{2} + \alpha^{2}}} d^{D}(k) = \\
    & = \hbar \lambda \frac{a(t)}{\sqrt{k^{2} + \alpha^{2}}}
\end{split}
\end{equation}

Решение уравнение Гейзенберга для оператора $a$:
\begin{equation}\notag
\begin{split}
    i \hbar \frac{\partial a(t)}{\partial t} & = [a(t), H] = \\
    & = \hbar \omega^{'}_a a(t) a^{\dagger}(t) a(t) + \hbar \int \omega^{'}_{R}(k) a(t) b^{\dagger}(k,t) b(k,t) d^{D}(k) + \\
    & + \hbar \lambda \int \frac{(a(t) b^{\dagger}(k,t) a(t) + a(t) a^{\dagger}(t) b(k,t))}{\sqrt{k^{2} + \alpha^{2}}} d^{D}(k) - \\
    & - \hbar \omega^{'}_a a^{\dagger}(t) a(t) a(t) - \hbar \int \omega^{'}_{R}(k) b^{\dagger}(k,t) b(k,t) a(t) d^{D}(k) - \\
    & - \hbar \lambda \int \frac{(b^{\dagger}(k,t) a(t) a(t) + a^{\dagger}(t) b(k,t) a(t))}{\sqrt{k^{2} + \alpha^{2}}} d^{D}(k) = \\
    & = xx' + yy' + zz' = \hbar \omega^{'}_a a(t) + \hbar \lambda \int \frac{b(k,t)}{\sqrt{k^{2} + \alpha^{2}}} d^{D}(k)
\end{split}
\end{equation}

где
\begin{equation}\notag
\begin{split}
    xx' & = \hbar \omega^{'}_a a(t) a^{\dagger}(t) a(t) - \hbar \omega^{'}_a a^{\dagger}(t) a(t) a(t) = \\
    & = \hbar \omega^{'}_a \underbrace{[a(t), a^{\dagger}(t)]}_{1} a(t)
    + \cancel{\hbar \omega^{'}_a a^{\dagger}(t) a(t) a(t)}
    - \cancel{\hbar \omega^{'}_a a^{\dagger}(t) a(t) a(t)} = \\
    & = \hbar \omega^{'}_a a(t) \\
    yy' & = \hbar \int \omega^{'}_{R}(k) \overbrace{a(t) b^{\dagger}(k,t) b(k,t)}^{= b^{\dagger}(k,t) a(t) b(k,t)} d^{D}(k) \\
    & - \hbar \int \omega^{'}_{R}(k) \overbrace{b^{\dagger}(k,t) b(k,t) a(t)}^{= b^{\dagger}(k,t) a(t) b(k,t)} d^{D}(k) = 0 \\
    zz' & = \hbar \lambda \int \frac{(a(t) b^{\dagger}(k,t) a(t)
    + a(t) a^{\dagger}(t) b(k,t))}{\sqrt{k^{2} + \alpha^{2}}} d^{D}(k) - \\
    & - \hbar \lambda \int \frac{(b^{\dagger}(k,t) a(t) a(t)
    + a^{\dagger}(t) b(k,t) a(t))}{\sqrt{k^{2} + \alpha^{2}}} d^{D}(k) = \\
\end{split}
\end{equation}
\begin{equation}\notag
\begin{split}
    zz' & = \hbar \lambda \int \frac{\overbrace{[a(t), b^{\dagger}(k,t)]}^{0} a(t) + \cancel{b^{\dagger}(k,t) a(t) a(t)}}{\sqrt{k^{2} + \alpha^{2}}} d^{D}(k) + \\
    & + \hbar \lambda \int \frac{[a(t), a^{\dagger}(t)] b(k,t) + \cancel{a^{\dagger}(t) a(t) b(k,t)}}{\sqrt{k^{2} + \alpha^{2}}} d^{D}(k) - \\
    & - \hbar \lambda \int \frac{\cancel{b^{\dagger}(k,t) a(t) a(t)} - a^{\dagger}(t) \overbrace{[a(t), b(k,t)]}^{0} +
    \cancel{a^{\dagger}(t) a(t) b(k,t)} }{\sqrt{k^{2} + \alpha^{2}}} d^{D}(k) = \\
    & = \hbar \lambda \int \frac{[a(t), a^{\dagger}(t)] b(k,t)}{\sqrt{k^{2} + \alpha^{2}}} d^{D}(k) =
    \hbar \lambda \int \frac{b(k,t)}{\sqrt{k^{2} + \alpha^{2}}} d^{D}(k)
\end{split}
\end{equation}


В результате, мы имеем следующий вид операторных уравнений для операторов уничтожения $a$ и $b$

{\large
\begin{equation}\label{syst}
\begin{cases}
    & \dot{a}(t) = -i \omega^{'}_a a(t) - i \lambda \int \frac{b(k,t)}{\sqrt{k^{2} + \alpha^{2}}} d^{D}(k) \\
    & \dot{b}(k,t) = -i \omega^{'}_{R}(k) b(k,t) -i \lambda \frac{a(t)}{\sqrt{k^{2} + \alpha^{2}}}
\end{cases}
\end{equation}}


Решение системы ищется адиабатически исключая резервуар.
То есть, в начале решается второе уравнение системы \eqref{syst} на ${b}(k,t) $, а затем его решение
подставляется в первое уравнение на $\dot{a}(t)$.
Физически это означает, что сперва формируется резервуар, а лишь затем он формирует свойства агента.
Уравнение на ${b}(k,t)$ является неоднородным дифференциальным уравнением, решение которого имеет
вид суммы соотвествующего линейного однородного дифференциального уравнения (ЛОДУ) и частоного решения
неоднородного дифференциального уравнения (ЛНДУ).
Решение ЛОДУ имеет вид:
\begin{equation}\label{LODY}
\begin{split}
    & \frac{\partial b(k,t)}{\partial t} = -i \omega^{'}_{R}(k) b(k,t); \\
    & b(k,t) = Ce^{-i \omega^{'}_{R}(k) t} \\
\end{split}
\end{equation}

В общем случае $C = C(t)$ и для определения данной функции необходимо подставить \eqref{LODY} в
соотвествующее ЛНДУ, таким образом:

\begin{equation}\notag
    (C(t)e^{-i \omega^{'}_{R}(k) t})^{'}_{t} = -i \omega^{'}_{R}(k) C(t)e^{-i \omega^{'}_{R}(k) t} -i \lambda \frac{a(t)}{\sqrt{k^{2} + \alpha^{2}}}
\end{equation}
\begin{equation}\label{LNDY}
    C(t) = -i \lambda \int \frac{a(t)e^{i \omega^{'}_{R}(k) t}}{\sqrt{k^{2} + \alpha^{2}}} dt + C_{1}
\end{equation}

Объединяя \eqref{LODY} и \eqref{LNDY} получим решение оператора ${b}(k,t)$ в следующем виде:

\begin{equation}\notag
b(k,t) = b(k)e^{-i \omega^{'}_{R}(k) t} -i \lambda \int_{0}^{t} \frac{a(\tau)e^{-i \omega^{'}_{R}(k) (t-\tau)}}{\sqrt{k^{2} + \alpha^{2}}} d\tau \\
\end{equation}
Данное решение легко проверить, подвставив его в \eqref{syst}.
\begin{multline}\notag
\Biggl( b(k)e^{-i \omega^{'}_{R}(k) t} -i \lambda \int_{0}^{t} \frac{a(\tau)e^{-i \omega^{'}_{R}(k) (t-\tau)}}{\sqrt{k^{2} + \alpha^{2}}} d\tau \Biggr)^{'}_{t} =  \\
= b(k) \Bigl( e^{-i \omega^{'}_{R}(k) t} \Bigr)^{'}_{t} - i \lambda \Biggl( \int_{0}^{t} \frac{a(\tau)e^{-i \omega^{'}_{R}(k) (t-\tau)}}{\sqrt{k^{2} + \alpha^{2}}} d\tau \Biggr)^{'}_{t} = \\
= -i \omega^{'}_{R}(k) \underbrace{b(k)e^{-i \omega^{'}_{R}(k) t}}_{b(k,0) = b(k,t)} -i \lambda \frac{a(t) \cancel{e^{-i \omega^{'}_{R}(k) (t-t)}}}{\sqrt{k^{2} + \alpha^{2}}} = \\
= -i \omega^{'}_{R}(k) b(k,t) -i \lambda \frac{a(t)}{\sqrt{k^{2} + \alpha^{2}}}
\end{multline}


Мы принимаем $\omega^{'}_{R}(k) = \omega k$ опираясь на известное волновое соотношение в оптике.
С учетом того, что аналитически решение интеграла имеет вид:
\begin{equation}\notag
    \int \frac{e^{-i \omega k (t - \tau)}}{k^{2} + \alpha^{2}} dk = \frac{\pi}{\alpha} e^{-\alpha \omega (t - \tau)},
\end{equation}
то в адиабатическом приближении решение системы \eqref{syst} имеет вид:
\begin{equation}\label{a}
    \dot{a}(t) =
    -i \omega^{'}_{a} a(t)
    -i \lambda \int \frac{b(k)e^{-i \omega k t}}{\sqrt{k^{2} + \alpha^2}} dk
    -\frac{\lambda^{2} \pi}{\alpha} \int_{0}^{t} a(\tau) e^{- \alpha \omega (t - \tau)} d\tau
\end{equation}

Решить данное интегрально-операторное уравнение явно не получится из-за специфики данного уравнения,
поэтому дальнейший анализ модели осуществлялся численно.
В работе использовался итерационный метод для решения данного уравнения и поиска наблюдаемой
величина $n_a=\langle{a^{\dagger}(t)a(t)\rangle}$, где усреденение проводится по когнитивным состониям ПР.
Для численного решения переведем интегральное уравнение в дифференциальное, таким образом:
\begin{equation}\label{dot_na}
\frac{\partial n_{a}(t)}{\partial t}
= \frac{\partial (a^{\dagger}(t) a(t))}{\partial t}
= \frac{\partial a^{\dagger}(t)}{\partial t}a(t) + a^{\dagger}(t)\frac{\partial a(t)}{\partial t}
= \dot{a}^{\dagger}(t) a(t) + a^{\dagger}(t) \dot{a}(t).
\end{equation}
Используя уравнение \eqref{a} мы можем получить следующие соотношения:
\begin{multline}\notag
    \dot{a}^{\dagger}(t) a(t)=
    i \omega^{'}_{a} a^{\dagger}(t) a(t)
    + i \lambda \int \frac{b^{\dagger}(k) a(t) e^{i \omega k t}}{\sqrt{k^{2} + \alpha^2}} dk \\
    -\frac{\lambda^{2} \pi}{\alpha} \int_{0}^{t} a^{\dagger}(\tau) a(t) e^{- \alpha \omega (t - \tau)} d\tau
\end{multline}
\begin{multline}\notag
    a^{\dagger}(t) \dot{a}(t) =
    -i \omega^{'}_{a} a^{\dagger}(t) a(t)
    -i \lambda \int \frac{a^{\dagger}(t) b(k)e^{-i \omega k t}}{\sqrt{k^{2} + \alpha^2}} dk \\
    - \frac{\lambda^{2} \pi}{\alpha} \int_{0}^{t} a^{\dagger}(t) a(\tau) e^{- \alpha \omega (t - \tau)} d\tau
\end{multline}
Подстовляя данные выражения в исходное соотношение \eqref{dot_na} можем получить дифференциальное уравнение наблюдаемой величины $n_a(t)$ в следующем виде:
\begin{multline}
    \frac{\partial n_{a}(t)}{\partial t} =
    i \lambda \int \frac{(b^{\dagger}(k) a(t) e^{i \omega k t} - a^{\dagger}(t) b(k)e^{-i \omega k t})}{\sqrt{k^{2} + \alpha^2}} dk - \\
    - \frac{\lambda^{2} \pi}{\alpha} \int_{0}^{t} e^{- \alpha \omega (t - \tau)} \overbrace{(a^{\dagger}(\tau) a(t) + a^{\dagger}(t) a(\tau))}^{n_{a}(t) \delta(t-\tau) + n_{a}(t) \delta(t-\tau)} d\tau
\end{multline}
в простом виде выражение \eqref{dot_na} можно переписать следующим образом:
\begin{equation}\label{na}
    \frac{\partial n_{a}(t)}{\partial t} =
    i \lambda \int \frac{(b^{\dagger}(k) a(t) e^{i \omega k t} - a^{\dagger}(t) b(k) e^{-i \omega k t})}{\sqrt{k^{2} + \alpha^2}} dk
    - 2 \frac{\lambda^{2} \pi}{\alpha} n_{a}(t)
\end{equation}

Данное выражение позволяет отследить как динамику процесса принятие решение у агента.
Третье слогаемое данного выражение имеет смысл ассимтотической стабилизации, при которой $n_a(t)$ уменьшается со временем.
В этом случае исключая второе слогаемое мы получим результат описанный в работе
$n_a(t)=n_{a}e^{-\frac{-2\lambda^2\pi}{\alpha}t}$ описывающий изночально пустой резервуар.
Второе слагаеме в выражении \eqref{na} описывает немарковское воздействие резервуара на агента ПР.
Для численного анализа данного слогаемого необходимо определить $(b^{\dagger}(k) a(t)$ и $a^{\dagger}(t) b(k)$.
Для их нахождения используем \eqref{a}, в этом случае выражение имеет вид:
\begin{multline}\label{bcrossk_dota}
    \frac{\partial (b^{\dagger}(k) a(t))}{\partial t} =
    b^{\dagger}(k) \dot{a}(t) =
    -i \omega^{'}_{a} b^{\dagger}(k) a(t) - \\
    -i \lambda \int \frac{b^{\dagger}(k) b(k) e^{-i \omega k t}}{\sqrt{k^{2} + \alpha^2}} dk
    -\frac{\lambda^{2} \pi}{\alpha} \int_{0}^{t} b^{\dagger}(k) a(\tau) e^{- \alpha \omega (t - \tau)} d\tau
\end{multline}
\begin{multline}\label{dota_bk}
    \frac{\partial (a^{\dagger}(t) b(k))}{\partial t} =
    \dot{a}^{\dagger}(t) b(k) =
    i \omega^{'}_{a} a^{\dagger}(t) b(k) + \\
    + i \lambda \int \frac{b^{\dagger}(k) b(k) e^{i \omega k t}}{\sqrt{k^{2} + \alpha^2}} dk
    - \frac{\lambda^{2} \pi}{\alpha} \int_{0}^{t} a^{\dagger}(\tau) b(k) e^{- \alpha \omega (t - \tau)} d\tau
\end{multline}
принимая во внимание коммутационные соотношения, получим следующие интегрально-дифференциальные уравнения:
\begin{equation}\label{bcrossk_dota_less}
    b^{\dagger}(k) \dot{a}(t) =
    -i \omega^{'}_{a} b^{\dagger}(k) a(t)
    -i \lambda \frac{n_{b} e^{-i \omega k t}}{\sqrt{k^{2} + \alpha^2}}
    -\frac{\lambda^{2} \pi}{\alpha} \int_{0}^{t} b^{\dagger}(k) a(\tau) e^{- \alpha \omega (t - \tau)} d\tau
\end{equation}
\begin{equation}\label{dota_bk_less}
    \dot{a}^{\dagger}(t) b(k) =
    i \omega^{'}_{a} a^{\dagger}(t) b(k)
    + i \lambda \frac{n_{b} e^{i \omega k t}}{\sqrt{k^{2} + \alpha^2}}
    - \frac{\lambda^{2} \pi}{\alpha} \int_{0}^{t} a^{\dagger}(\tau) b(k) e^{- \alpha \omega (t - \tau)} d\tau
\end{equation}

Используя конечно-разностную схему второго порядка $f'(x) = \lim_{\Delta x \to 0} \frac{f(x + \Delta x) - f(x)}{\Delta x}$
можно записать полученные выражения \eqref{na}, \eqref{bcrossk_dota_less} и \eqref{dota_bk_less} в следующем виде:
\begin{multline}
    n_{a}(t + \Delta t) =
    \Biggl[i \lambda \int \frac{R_{1}(k,t) e^{i \omega k t} - R_{2}(k,t) e^{-i \omega k t}}{\sqrt{k^{2} + \alpha^2}} dk - \\
    - 2 \frac{\lambda^{2} \pi}{\alpha} n_{a}(t) \Biggr] \Delta t + n_{a}(t)
\end{multline}
\begin{multline}
    R_{1}(k,t + \Delta t) =
    \Biggl[-i \omega^{'}_{a} R_{1}(k,t)
    -i \lambda \frac{n_{b} e^{-i \omega k t}}{\sqrt{k^{2} + \alpha^2}} - \\
    -\frac{\lambda^{2} \pi}{\alpha} \int_{0}^{t} R_{1}(k,\tau) e^{- \alpha \omega (t - \tau)} d\tau \Biggr] \Delta t + R_{1}(k,t)
\end{multline}
\begin{multline}
    R_{2}(k,t + \Delta t) =
    \Biggl[i \omega^{'}_{a} R_{2}(k,t)
    + i \lambda \frac{n_{b} e^{i \omega k t}}{\sqrt{k^{2} + \alpha^2}} - \\
    - \frac{\lambda^{2} \pi}{\alpha} \int_{0}^{t} R_{2}(k,\tau) e^{- \alpha \omega (t - \tau)} d\tau \Biggr] \Delta t + R_{2}(k,t)
\end{multline}
где  $R_{1}(k,t) = b^{\dagger}(k) a(t)$, $R_{2}(k,t) = a^{\dagger}(t) b(k)$.
Решение поставленной задачи сводится к решению трех уравнений описанных выше. Их решение позволит отследить динамику ПР у агента и влияние структуры резервуара
на принятие решения.