%! Author = jaroslav
%! Date = 22.04.20
%\begin{titlepage}
%    \tiny
%    \scriptsize
%    \footnotesize
%    \small
%    \normalsize
%    \large
%    \Large
%    \LARGE
%    \huge
%    \Huge
%    \begin{center}
%        \large
%        РЕФЕРАТ
%    \end{center}
%
%    Отчет \total{page} страницы, %\total{mequation} формул,
%    \total{mfigure} рисунков,
%    \total{mtable} таблиц, \total{bibcnt} источников,
%
%    \hspace{0.5cm}ТЕОРИЯ ПРИНЯТИЯ РЕШЕНИЙ, ЭКСПЕРИМЕНТ ЭЛЛСБЕРГА,
%
%    \hspace{0.5cm}ОШИБКА КОНЪЮНКЦИИ, ЭФФЕКТ ДИЗЪЮНКЦИИ,
%
%    \hspace{0.5cm}ИНТЕРНЕТ РЕСУРСЫ, ДИЛЕММА ЗАКЛЮЧЕННОГО
%
%    Объектом исследования является модель социально-значимых Интернет
%    ресурсов, которое рассматривает отношения между индивидами в рамках
%    дискусси на популярном Интернет площадке.
%
%    Целью данной работы является поиск связей между моделью социально-
%    значимых Интернет ресурсов и экспериментами, объяснение которых
%    простыми моделями не представляется возможным.
%    Выявление этих связей позволит узнать возможные пути объяснения
%    этих эффектов, а также предсказать возможные другие когнитивные
%    эффекты.
%
%    В результате работы была вывлена связь между экспериментами и
%    моделью социально-значимых Интернет ресурсов
%
%    В дальнейшем, эта работа позвоит выявить связь между моделью
%    социально-значимых Интернет ресусов и квантово-подобным моделями
%    приянтия решений в условиях неопределенности
%\end{titlepage}