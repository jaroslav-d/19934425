\Conclusion % заключение к отчёту

В ходе проведенной работы было изучено большинство когнитивных эффектов возникающие в условиях неопределенности.
На основании изученных работ, была разработана модель принятия решения агентом с учетом его взаимодействия
(обмена информацией) с внешним окружением, описываемым резервуаром.
Когнитивное состояние агента в рассматриваемой модели описывается вектором в Гильбертовом пространстве,
а процесс ПР в представлении вторичного квантования характеризуется операторами рождения $a^{\dagger}$
и уничтожение $a$, которые определяют динамику ПР.
Наблюдаемой величиной в рассматриваемой задаче является оператор принятия решения $n_{a} = a^{\dagger} a$,
который усреднялся по когнитивным состояниям ПР и определяет основную динамику данного процесса.
В зависимости от типа внешнего информационного резервуара процесс принятия решения агентом имел марковский
и немарковский характер.
Проведен аналитический и численный анализ модели динамического принятия решения агентом с учетом как
марковости, так и не марковости резервуара, имеющего бесконечное число степеней свободы.
Было показано, что влияние резервуара на агента ПР приводит к различным режимам затухания вероятности
принятия решения, имеющим экспоненциальный~характер в случае марковости процесса и осциллирующий
характер~в~случае немарковости процесса.
Переход от марковского к немарковскому~процессу осуществляется через изменение параметров ядра
взаимодействия $g(k)$ агента ПР и внешнего информационного окружения.
Был проведен анализ зависимости динамики принятия решения агентом от вводимых параметров процесса,
который позволил выявить, что является режимом существенно~осциллирующего немарковоского взаимодействия
агента ПР с резервуаром.
Так как немарсковский процесс имеет сложную по своей структуре динамику, в работе был проведен аналитический
анализ полученных результатов, который позволил получить аналитическое решение рассматриваемой модели.
Данное аналитическое решение хорошо согласуется с полученным численный моделированием модели.

%%% Local Variables: 
%%% mode: latex
%%% TeX-master: "rpz"
%%% End: 
