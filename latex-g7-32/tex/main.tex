\documentclass[%
master,      % тип документа
natbib,      % использовать пакет natbib для "сжатия" цитирований
subf,        % использовать пакет subcaption для вложенной нумерации рисунков
href,        % использовать пакет hyperref для создания гиперссылок
colorlinks,  % цветные гиперссылки
%fixint,     % включить прямые знаки интегралов
]{disser}

\usepackage[
a4paper, mag=1000,
left=3cm, right=1cm, top=2cm, bottom=2cm, headsep=0.7cm, footskip=1cm
]{geometry}

\usepackage{natbib}
\usepackage[intlimits]{amsmath}
\usepackage{amssymb,amsfonts}

\usepackage[T2A]{fontenc}
\usepackage[utf8]{inputenc}
\usepackage[english,russian]{babel}
\ifpdf\usepackage{epstopdf}\fi
\usepackage[autostyle]{csquotes}

\usepackage[singlelinecheck=false]{caption}
\usepackage{cmap}
\usepackage{cancel}

%\usepackage[document]{ragged2e}
\pretolerance=10000
%\sloppy

% Шрифт Times в тексте как основной
%\usepackage{tempora}
% альтернативный пакет из дистрибутива TeX Live
\usepackage{cyrtimes}

% Шрифт Times в формулах как основной
%\usepackage[varg,cmbraces,cmintegrals]{newtxmath}
% альтернативный пакет
\usepackage[lite]{mtpro2}
\pdfmapfile{mtpro2.map}

% Список сокращений и условных обозначений
\usepackage[intoc,nocfg,russian]{nomencl}
\newcommand{\nomencl}[2]{#1 --- #2\nomenclature{#1}{#2}}
\setlength{\nomlabelwidth}{3em}
\setlength{\nomitemsep}{-\parsep}
\renewcommand{\nomlabel}[1]{#1 ---}
\makenomenclature

% Плавающие рисунки "в оборку".
\usepackage{wrapfig}

% Номера страниц снизу и по центру
\pagestyle{footcenter}
\chapterpagestyle{footcenter}

% Точка с запятой в качестве разделителя между номерами цитирований
%\setcitestyle{semicolon}

% Использовать полужирное начертание для векторов
\let\vec=\mathbf

% Включать подсекции в оглавление
\setcounter{tocdepth}{2}

% Подключить пакет для вставки кода в файл
\usepackage{listings}

\definecolor{codegreen}{rgb}{0,0.6,0}
\definecolor{codegray}{rgb}{0.5,0.5,0.5}
\definecolor{codepurple}{rgb}{0.58,0,0.82}
\definecolor{backcolour}{rgb}{0.95,0.95,0.92}

\lstset{ %
language=Python,                 % выбор языка для подсветки (здесь это С)
%basicstyle=\small\sffamily, % размер и начертание шрифта для подсветки кода
basicstyle=\tiny, % размер и начертание шрифта для подсветки кода
numbers=left,               % где поставить нумерацию строк (слева\справа)
%numberstyle=\tiny,           % размер шрифта для номеров строк
stepnumber=1,                   % размер шага между двумя номерами строк
numbersep=5pt,                % как далеко отстоят номера строк от подсвечиваемого кода
backgroundcolor=\color{white}, % цвет фона подсветки - используем \usepackage{color}
commentstyle=\color{codegreen},
keywordstyle=\color{magenta},
numberstyle=\color{codegray},
stringstyle=\color{codepurple},
showspaces=false,            % показывать или нет пробелы специальными отступами
showstringspaces=false,      % показывать или нет пробелы в строках
showtabs=false,             % показывать или нет табуляцию в строках
frame=single,              % рисовать рамку вокруг кода
tabsize=2,                 % размер табуляции по умолчанию равен 2 пробелам
captionpos=t,              % позиция заголовка вверху [t] или внизу [b]
breaklines=true,           % автоматически переносить строки (да\нет)
breakatwhitespace=false, % переносить строки только если есть пробел
escapeinside={#}   % если нужно добавить комментарии в коде
}

\usepackage{graphicx}
\graphicspath{{pictures/}}

\captionsetup[figure]{justification=centering}
\captionsetup[table]{position=top,aboveskip=5pt}

% Покдлючить пакет с возможностью подсчета станиц, рисунков, формул, таблиц и т.д.
\usepackage{totcount}
\regtotcounter{page}
\newtotcounter{mfigure}
\newtotcounter{mequation}
\newtotcounter{mtable}
\newtotcounter{bibcnt}

\def\oldfigure{} \let\oldfigure=\figure
\def\figure{\stepcounter{mfigure}\oldfigure}

\def\oldequation{} \let\oldequation=\equation
\def\equation{\stepcounter{mequation}\oldequation}

\def\oldtable{} \let\oldtable=\table
\def\table{\stepcounter{mtable}\oldtable}

\def\oldbibitem{} \let\oldbibitem=\bibitem
\def\bibitem{\stepcounter{bibcnt}\oldbibitem}

\makeatletter
\setlength{\parindent}{7.5ex}
\renewcommand{\tocprethechapter}{}
\renewcommand{\tocpostthechapter}{}
\renewcommand{\tocpostthesection}{}
\renewcommand{\postthesection}{\ }
\renewcommand{\tocsectionindent}{0em}
\renewcommand{\tocsectionnameindent}{1.5em}
\renewcommand{\sectionindent}{10em}
\renewcommand{\captionlabeldelim}{\textendash}
\renewcommand{\thetable}{\@arabic\c@table}
\renewcommand{\theequation}{\@arabic\c@equation}
\renewcommand{\fnum@figure}{Рисунок \thefigure}
\renewcommand{\labelitemi}{\textendash}
\makeatother

\DeclareCaptionFormat{plain}{\captionlabelfont #1 \captionlabeldelim \ #3 \par}

\makeatletter
\renewcommand*\l@chapter[2]{%
\ifnum \c@tocdepth >\m@ne
%\addpenalty{-\@highpenalty}
\vskip 1.0em \@plus\p@
%\setlength\@tempdima{4.5em}
\begingroup
\parindent \z@ \rightskip \@pnumwidth
\parfillskip -\@pnumwidth
%\leavevmode\tocchapterfont
\advance\leftskip\@tempdima
\hskip -\leftskip
#1\nobreak
\tocchapterfillfont\tocchapterfill\hfill
\nobreak\hb@xt@\@pnumwidth{\hss\tocchapternumfont #2}\par
\penalty\@highpenalty
\endgroup
\fi}
\makeatother

\makeatletter
\renewcommand{\@makechapterhead}[1]{% Начало макроопределения
\vspace*{20 pt}% Пустое место вверху страницы
{\parindent=0pt
\hspace{1.5 cm}
\upshape \mdseries \rmfamily \normalsize \large
\thechapter \hspace{2 pt} #1\par % заголовок главы
\nopagebreak
\vspace{5 pt}
}
}
\makeatother

\renewcommand{\sectionfont}{\upshape \mdseries \rmfamily \normalsize \large}
\renewcommand{\sectionindent}{1.6 cm}

\begin{document}

    % Переопределение стандартных заголовков
    \def\introname{\upshape \mdseries \rmfamily \normalsize \large ВВЕДЕНИЕ}
    \def\contentsname{\upshape \mdseries \rmfamily \normalsize \large СОДЕРЖАНИЕ}
    \def\conclusionname{\upshape \mdseries \rmfamily \normalsize \large ЗАКЛЮЧЕНИЕ}
    \def\bibname{\upshape \mdseries \rmfamily \normalsize \large СПИСОК ИСПОЛЬЗОВАННЫХ ИСТОЧНИКОВ}

    %
    % Титульный лист на русском языке
    %

%    \institution{Название организации}
%
%    % Имя лица, допускающего к защите (зав. кафедрой)
%    \apname{ФИО зав. кафедрой}
%
%    \title{ДИССЕРТАЦИЯ\\[-14pt]на соискание ученой степени\\МАГИСТРА}
%
%    \topic{Тема диссертации}
%
%    % Автор
%    \author{ФИО автора}
%    % Группа
%    \group{1111/1}
%    % Номер направления
%    \coursenum{111111}
%    \course{Название направления}
%    % Номер магистерской программы
%    \masterprognum{111111}
%    \masterprog   {Название программы}
%
%    % Научный руководитель
%    \sa      {ФИО руководителя}
%    \sastatus{д.~ф.-м.~н., ст.~н.~с.}
%    % Второй научный руководитель
%    %\sasnd      {ФИО руководителя}
%    %\sasndstatus{д.~ф.-м.~н., ст.~н.~с.}
%
%    % Рецензент
%    \rev      {ФИО рецензента}
%    \revstatus{д.~ф.-м.~н., в.~н.~с.}
%    % Второй рецензент
%    %\revsnd      {ФИО рецензента}
%    %\revsndstatus{д.~т.~н., ст.~н.~с.}
%
%    % Консультант
%    \con{ФИО консультанта}
%    \conspec{вопросам\\охраны труда}
%    \constatus{к.~т.~н., доц.}
%    % Второй консультант
%    %\consnd{ФИО консультанта}
%    %\consndspec{экономическим\\вопросам}
%    %\consndstatus{к.~э.~н., доц.}
%
%    % Город и год
%    \city{Санкт-Петербург}
%    \date{\number\year}
%
%    \maketitle

    %%
    %% Titlepage in English
    %%
    %
    %\institution{Name of Organization}
    %
    %% Approved by
    %\apname{Professor S.\,S.~Sidorov}
    %
    %\title{Master's Thesis}
    %
    %% Topic
    %\topic{Dummy Title}
    %
    %% Author
    %\author{Author's Name} % Full Name
    %\course{Physics} % Specialization
    %
    %\group{} % Study Group
    %\masterprog   {Title of program}
    %
    %% Scientific Advisor
    %\sa       {I.\,I.~Ivanov}
    %\sastatus {Professor}
    %
    %% Reviewer
    %\rev      {P.\,P.~Petrov}
    %\revstatus{Associate Professor}
    %
    %% Consultant
    %\con{}
    %\conspec{}
    %\constatus{}
    %
    %% City & Year
    %\city{Saint Petersburg}
    %\date{\number\year}
    %
    %\maketitle[en]

    % Реферат для НИР
%    %! Author = jaroslav
%! Date = 22.04.20
%\begin{titlepage}
%    \tiny
%    \scriptsize
%    \footnotesize
%    \small
%    \normalsize
%    \large
%    \Large
%    \LARGE
%    \huge
%    \Huge
%    \begin{center}
%        \large
%        РЕФЕРАТ
%    \end{center}
%
%    Отчет \total{page} страницы, %\total{mequation} формул,
%    \total{mfigure} рисунков,
%    \total{mtable} таблиц, \total{bibcnt} источников,
%
%    \hspace{0.5cm}ТЕОРИЯ ПРИНЯТИЯ РЕШЕНИЙ, ЭКСПЕРИМЕНТ ЭЛЛСБЕРГА,
%
%    \hspace{0.5cm}ОШИБКА КОНЪЮНКЦИИ, ЭФФЕКТ ДИЗЪЮНКЦИИ,
%
%    \hspace{0.5cm}ИНТЕРНЕТ РЕСУРСЫ, ДИЛЕММА ЗАКЛЮЧЕННОГО
%
%    Объектом исследования является модель социально-значимых Интернет
%    ресурсов, которое рассматривает отношения между индивидами в рамках
%    дискусси на популярном Интернет площадке.
%
%    Целью данной работы является поиск связей между моделью социально-
%    значимых Интернет ресурсов и экспериментами, объяснение которых
%    простыми моделями не представляется возможным.
%    Выявление этих связей позволит узнать возможные пути объяснения
%    этих эффектов, а также предсказать возможные другие когнитивные
%    эффекты.
%
%    В результате работы была вывлена связь между экспериментами и
%    моделью социально-значимых Интернет ресурсов
%
%    В дальнейшем, эта работа позвоит выявить связь между моделью
%    социально-значимых Интернет ресусов и квантово-подобным моделями
%    приянтия решений в условиях неопределенности
%\end{titlepage}


\end{document}