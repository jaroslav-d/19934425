\Introduction

Работа начинается с обзора литературы, в котором рассказываются ключевые~парадоксы~в~принятии
решения индивидом в условиях неопределенности, рассматриваются существующие модели принятия решений,
раскрываются необходимые понятия из квантовой физики для объяснения последующих рассуждений в
основной работе.
Актуальность темы заключается в разрешении парадоксов когнитивных и смежных наук в задачах принятия
решений с помощью квантовых подходов.
Работа заканчивается результатами численного~моделирования принятия решения человеком в условиях
неопределенности, находящимся в информационной среде.
Актуальность работы заключается в пересмотре существующих квантово-подобных моделей принятия решений
в рамках немарковских процессов поведения для учета эффектов случайности мышления.
Целью работы является разработка немарковской динамической квантово-подобной модели принятия решений
в условиях неопреленности.
Задачи:
\begin{itemize}
    \item Разработать модель принятия решений с учетом немарковского взаимодействия агента с информационым окружением(резервуаром)
    \item Исследовать влияние структуры резервуара на принятие решения агентом
    \item Сравнить полученные результаты с существующими моделями
\end{itemize}
В работе были использованы методы интерационного решения, методы решения динейных дифференциальных уравнений,
методы дискретной математики, методы поиска аналитического решения.
В работе было исследовано влияние структуры резервуара на принятие решения агентом в условиях неопределенности,
были найдены граничные условия перехода разработанной модели к существующим.
